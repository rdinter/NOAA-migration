\documentclass[]{article}
\usepackage{lmodern}
\usepackage{setspace}
\setstretch{2}
\usepackage{amssymb,amsmath}
\usepackage{ifxetex,ifluatex}
\usepackage{fixltx2e} % provides \textsubscript
\ifnum 0\ifxetex 1\fi\ifluatex 1\fi=0 % if pdftex
  \usepackage[T1]{fontenc}
  \usepackage[utf8]{inputenc}
\else % if luatex or xelatex
  \ifxetex
    \usepackage{mathspec}
  \else
    \usepackage{fontspec}
  \fi
  \defaultfontfeatures{Ligatures=TeX,Scale=MatchLowercase}
\fi
% use upquote if available, for straight quotes in verbatim environments
\IfFileExists{upquote.sty}{\usepackage{upquote}}{}
% use microtype if available
\IfFileExists{microtype.sty}{%
\usepackage{microtype}
\UseMicrotypeSet[protrusion]{basicmath} % disable protrusion for tt fonts
}{}
\usepackage[margin=1in]{geometry}
\usepackage{hyperref}
\hypersetup{unicode=true,
            pdftitle={The Effect of Disasters on Migration},
            pdfauthor={Jonathan Eyer, Adam Rose, Noah Miller; and Robert Dinterman},
            pdfborder={0 0 0},
            breaklinks=true}
\urlstyle{same}  % don't use monospace font for urls
\usepackage{longtable,booktabs}
\usepackage{graphicx,grffile}
\makeatletter
\def\maxwidth{\ifdim\Gin@nat@width>\linewidth\linewidth\else\Gin@nat@width\fi}
\def\maxheight{\ifdim\Gin@nat@height>\textheight\textheight\else\Gin@nat@height\fi}
\makeatother
% Scale images if necessary, so that they will not overflow the page
% margins by default, and it is still possible to overwrite the defaults
% using explicit options in \includegraphics[width, height, ...]{}
\setkeys{Gin}{width=\maxwidth,height=\maxheight,keepaspectratio}
\IfFileExists{parskip.sty}{%
\usepackage{parskip}
}{% else
\setlength{\parindent}{0pt}
\setlength{\parskip}{6pt plus 2pt minus 1pt}
}
\setlength{\emergencystretch}{3em}  % prevent overfull lines
\providecommand{\tightlist}{%
  \setlength{\itemsep}{0pt}\setlength{\parskip}{0pt}}
\setcounter{secnumdepth}{0}
% Redefines (sub)paragraphs to behave more like sections
\ifx\paragraph\undefined\else
\let\oldparagraph\paragraph
\renewcommand{\paragraph}[1]{\oldparagraph{#1}\mbox{}}
\fi
\ifx\subparagraph\undefined\else
\let\oldsubparagraph\subparagraph
\renewcommand{\subparagraph}[1]{\oldsubparagraph{#1}\mbox{}}
\fi

%%% Use protect on footnotes to avoid problems with footnotes in titles
\let\rmarkdownfootnote\footnote%
\def\footnote{\protect\rmarkdownfootnote}

%%% Change title format to be more compact
\usepackage{titling}

% Create subtitle command for use in maketitle
\newcommand{\subtitle}[1]{
  \posttitle{
    \begin{center}\large#1\end{center}
    }
}

\setlength{\droptitle}{-2em}
  \title{The Effect of Disasters on Migration}
  \pretitle{\vspace{\droptitle}\centering\huge}
  \posttitle{\par}
  \author{Jonathan Eyer, Adam Rose, Noah Miller\footnote{University of Southern
  California} \\ and Robert Dinterman\footnote{The Ohio State University}}
  \preauthor{\centering\large\emph}
  \postauthor{\par}
  \predate{\centering\large\emph}
  \postdate{\par}
  \date{13 June 2017}


\begin{document}
\maketitle
\begin{abstract}
While post-disaster migration can move vulnerable peoples from dangerous
regions to relatively safe areas, little is known about the processes
through which migrants select new homes. We refine a spatial econometric
model of migrant flows to examine the characteristics of the
destinations of migrants leaving the New Orleans area following
Hurricane Katrina in 2005 and in 2006 when the pressure to evacuate was
lessened. We find that migrants in 2005 settled close to New Orleans,
with little consideration for destination amenities or characteristics.
After the immediate threat had dissipated, however, migrants were more
likely to consider economic characteristics of destinations.
\end{abstract}

\newpage

\section{Introduction}\label{introduction}

Natural disasters can cause widespread destruction and weaken local
economies. These impacts can lead to permanent migration away from
disaster-affected areas. Such permanent, or even semi-permanent,
migration induced by natural disasters has the potential to
significantly reshape the distribution of national and global
populations and economies (see e.g. McIntosh (2008)). Moreover, because
migration moves people out of the path of some disasters, and
potentially into the path of other ones, post-disaster migration has
implications for the risks associated with future events (See (Gráda and
O'Rourke 1997)). Finally, the migration itself and the loss of community
cohesion suggests the need for consideration of mental health support in
communities that will receive large numbers of disaster migrants (See
(Weber and Peek 2012)).

Migration following large disasters is well-documented after major
events like Hurricane Katrina, the 2011 Tohoku earthquake/tsunami and
subsequent Fukushima nuclear disaster, and the 2004 Indian Ocean
tsunami. Because of the pressures placed on the affected population,
disasters can cause migration among a wider portion of the population
than those who migrate normally (in a non-disaster related context) (See
(Gray et al. 2014)). While the propensity for disaster-affected
populations to migrate is documented, less is known about the
preferences that impact the destination of disaster-affiliated migrants.

Factors in migration decisions are generally framed in the context of
``push'' factors and ``pull'' factors. Push factors cause people to want
to leave the origin while pull factors cause people to want to go to a
specific destination. High unemployment in the origin might push people
to leave, for example. Similarly, a low cost of living might pull people
toward a particular destination.

The circumstances of a disaster, however, amy shift the relative
importance of various pull factors in attracting migrants. This could
occur either because preferences over these pull factors are dependent
on the state in which the decision to migrate is made. Alternatively,
the disaster induces traditional non-migrants to move and if these
people have different preferences than those who are traditional movers,
then the overall destination of post-disaster migrants will shift.

An understanding of these pull factors is important for crafting natural
disaster policies, understanding the likely evolution of disaster
damages, and evaluating the prospects for repatriation. For example, if
post-disaster migrants are credit constrained and unable to move to the
optimal location, government subsidies for relocation costs might be
justified. Similarly, if post-disaster migrants move to other areas that
are at high risk of natural disasters, future disaster losses may
actually increase following the migration.

Hurricane Katrina, which struck New Orleans in 2005, provides an ideal
case study to examine the factors that influence the destination of
disaster migrants. Most residents of New Orleans evacuated prior to the
Hurricane, and following the storm most remaining residents were
evacuated by the Federal Emergency Management Agency (FEMA).
Approximately 1.5 million people evacuated the New Orleans area. 96\% of
New Orleans residents and 80\% of residents surrounding the city
eventually left their homes (See (Groen and Polivka 2008; Elliott and
Pais 2006)). While a large number of evacuees were relocated to Houston
by FEMA, Katrina evacuees relocated throughout the country. Nearly every
state received FEMA funding for costs associated with supporting
evacuees from Katrina. Many of those who evacuated following Hurricane
Katrina never returned to the New Orleans area. These permanent migrants
were generally younger, more likely to have children, and more likely to
be black than those who returned to New Orleans (See (Groen and Polivka
2010)). There was also an increased flow of migrants from neighboring
communities in the years following Katrina compared to the years prior
to Katrina, indicating that those migrants who relocated to nearby
communities were more likely to return than those who relocated to
distant ones (See (Fussell, Curtis, and DeWaard 2014)).

In this paper, we examine the migration pull factors in terms of
characteristics of the destinations of post-Katrina migration out of the
New Orleans area by using data on the movement of IRS return filings
between counties and a range of county-destination attributes. This
paper contributes to the literature by estimating the relative
importance of a range of factors in post-disaster relocation decisions.
This work conveys a range of policy implications surrounding disasters
and climate change. By identifying the characteristics that draw
migrants following natural disasters, we inform future migration
patterns as disasters grow more frequent. Our consideration of distance
in the relocation decision also highlights the extent to which
post-disaster migrants will be removed from similarly disaster-prone
areas. Finally, we contribute to a small but growing set of studies that
model migration in an explicitly spatial econometric context.

The rest of the paper proceeds as follows. In Section \ref{theory}, we
review the theoretical structure of migration decisions, in Section
\ref{data} discuss our data sources, in Section \ref{sec:meth} present
our estimating equations, and in Section \ref{sec:results} present our
results.

\section{\texorpdfstring{Conceptual Underpinnings
\label{theory}}{Conceptual Underpinnings }}\label{conceptual-underpinnings}

From an economic standpoint, migration decisions are based on households
comparing their expected lifetime utility in their current location (the
origin) to a location to which they could move (the destination) (See
(Greenwood 1985; Greenwood 1975)). Yun and Waldorf (2016) examine the
decision about whether or not to migrate in an expected lifetime utility
framework and focus on the extent to which Katrina induced migration by
those who would not otherwise have migrated. The utility that a
household expects to receive from living in a particular location
depends on economic variables such as the wages and cost-of-living
associated with an area, but also on non-economic variables such as
environmental amenities, family and social ties, and perceptions about
safety. A household will decide to migrate if the increase in expected
lifetime utility obtained by moving from the origin to the destination
exceeds the costs associated with moving. These costs include the
financial costs associated with moving, as well as more abstract factors
such as the social costs incurred by the move.

The decision to migrate is generally endogenous to migrant
characteristics. Highly-skilled migrants who expect to receive large
wage premiums are more likely to migrate than low-skilled workers (See
Borjas (1987)). Similarly, migration is costly, Chiswick (1999){]} note
that those who are less credit-constrained are more able to afford the
upfront costs associated with an optimal relocation decision.

Natural disasters, however, cause exogenous variation in the expected
lifetime utility at the origin. For example, property damage would
require repair costs in order to stay at the origin, and a weakened
local economy would lower wages at the origin. Similarly, if a disaster
causes households to update their beliefs about the likelihood and
severity of subsequent events, this could lower the expected utility of
remaining in the origin. These effects would cause households to
re-evaluate their location decisions and potentially choose to migrate
due to the decreased expected life-time utility at their origin (See
(Yun and Waldorf 2016)).

In the event of major natural disasters like Hurricane Katrina and the
Fukushima nuclear disaster, the push factors are relatively obvious;
people leave the origin because of mandatory evacuation requirements,
legal inability to return due to quarantines, loss of employment
opportunities, etc. It is less obvious what draws migrants to particular
locations following a disaster. One might be particularly concerned that
post-disaster migrants are systematically different than those who
choose to migrate under other circumstances. Disaster-related migrants,
for example, might feel compelled to relocate more quickly or have less
wealth with which to bear moving costs. Hence, they may not move to
optimal locations in comparison to normal circumstances, or what Yun and
Waldorf (2016) refer to as ``double-victimization.'' Black et al. (2011)
suggest that population movements due to disasters are typically short
distance, though this conclusion seems to be counter to what happened in
the aftermath of Hurricane Katrina.

Several variables have been suggested, and some tested, to explain the
pull factors. Many of these are traditional in the gravity model
literature of migration, such as wage and cost-of-living differentials,
distance, moving costs, and general economic health of the destination
(See (Borjas 1987; Rupasingha, Liu, and Partridge 2015)). Broadening the
analysis leads to consideration of amenities, family ties, racial/ethnic
affinities, migration networks, and institutions (See (McKenzie and
Rapoport 2010; Nifo and Vecchione 2014)). The destination choice itself
is dependent on the reason that drives the individual to migrate (See
(Findlay 2011)). One might conclude that short or long-run hazard
vulnerability would be major considerations, but Black et al. (2011) and
Fielding (2011) emphasize the primacy of socioeconomic over
environmental variables in current migration decisions, though on the
basis of only anecdotal information.

\section{\texorpdfstring{Data \label{data}}{Data }}\label{data}

Our primary source of data is the Internal Revenue Service (IRS)
Statistics of Income Division's migration data. These data report the
flows of populations between counties based on changes in the location
from which tax returns are filed and the number of exemptions that are
claimed on those tax returns. The IRS reports both outflow migration as
well as inflow migration. The IRS data also include the number of
exemptions of people who do not move, providing a base-level population
value that is comparable to the migration data. In order to ensure the
privacy of individual filers, the IRS suppresses observations in which
fewer than ten filers migrated between an origin-destination pair. We
treat these values as true zeroes.

Given our focus on New Orleans, we restrict our interest to outflow
migration from the parishes most severely affected by Hurricane Katrina
in 2005. We define the population affected by Hurricane Katrina as those
residing in Cameron, Jefferson, Lafourche, Orleans, Plaquemines,
St.~Bernard, St.~Charles, St.~John the Baptist, St.~Tammany, and
Terrebonne Parishes. These parishes constitute the New Orleans metro, as
well as two surrounding Parishes (Lafourche and Terrebonne) which adjoin
the metro area.While there was some migration between affected regions
(i.e., moving from a county that was severely affected to one that was
slightly less affected), we remove these migrants from our sample to
facilitate a simpler interpretation of outflow migrants.
{[}\^{}parish{]}: While Louisiana is organized into parishes rather than
counties, we will use the term counties throughout this paper to
facilitate discussion of destination locations.

We aggregate annual migration flows to each destination county across
the five highly-affected origin counties between 2000 and 2010 The
result is an 11-year panel of population flows to the 3095 destination
counties.\footnote{There are 3,144 counties and county equivalents in
  the U.S. and affected counties are removed from the set of potential
  destination counties as well as any counties for which explanatory
  variables are unavailable.} There is a non-zero number of migrants to
approximately 5.4\% of the county-year observations in our dataset. In
2005, however, 13.8\% of US counties received migrants from the affected
area. With the exception of 2005, the number of migrants and the
proportion of counties that receive migrants from New Orleans is
relatively consistent over time.

Most migrants from the New Orleans area move to counties that are
relatively close. Table \textasciitilde{}\ref{tab:commondeststate} and
\textasciitilde{}\ref{tab:commondest} present the states and counties
that received the greatest proportion of migrants from the New Orleans
area in 2005. Each of the five states that received the most migrants
from New Orleans area in the South, as are the ten most common
destination counties. With the exception of Tangiphoa and Ascension
parishes in Louisiana, each of the most common destination counties
contain a metropolitan area.

We supplement the IRS migration data with a number of explanatory
variables that might affect the relative attractiveness of a destination
county. Annual county-level unemployment rates are obtained from the
Bureau of Labor Statistics (BLS). Average annual wage data are obtained
the the BLS' Quarterly Census of Wages. Median monthly rents are
obtained from the Department of Housing and Urban Development (HUD). For
each variable we merge these data with the IRS migration data by county
and year.

\section{\texorpdfstring{Methodology
\label{sec:meth}}{Methodology }}\label{methodology}

In order to understand how Hurricane Katrina affected migration
preferences, we estimate a series of models of migration outflow from
the affected counties. We specify the model

\begin{equation} \label{eq:basereg}
    mig_{jt} = \beta_0 + \beta_1 population_{jt} + \beta_2 distance_j + X_E + Year2005_t + \epsilon_{jt}
\end{equation}

where \(mig_{jt}\) is the number of total migrants that moved from the
parishes affected by Hurricane Katrina to county \(j\) in year \(t\),
\(population_{jt}\) is county \(j's\) population in year \(t\), and
\(distance_j\) is the Euclidean distance between the New Orleans area
and county \(j\). \(X_E\) is a matrix of economic explanatory variables,
composed of a county's unemployment rate, average annual wages, and
average monthly rent, and a dummy variable for whether or not a county
is in a metro area or not. Unemployment and average annual wages each
convey information about the labor market in a given county, while the
median rent proxies for the local cost-of-living. \(Year2005_t\) is an
indicator variable for whether or not an observation corresponds to
2005, accounting for the fact that many more migrants left the New
Orleans area in 2005 than in other years. This formulation provides a
baseline understanding of the preferences underlying migration decisions
away from the New Orleans area.

Next, we interact each of our explanatory variables with the
\(Year2005\) variable. The interaction terms capture the change in
preferences over each pull factor variable relative to the other years
in our sample, when a major disaster did not strike. If Hurricane
Katrina shifted the relative importance of pull factors in the
destination-selection process, we would expect these interaction terms
to be statistically significant.

We consider a set of dependent variables for our regressions, and
estimate the model separately with each potential dependent variable.
First, we consider the number of migrants to county \(j\) itself. Next,
we consider the inverse hyperbolic sine of migrants to county \(j\),
which is comparable to the natural log and yields semi-elasticities.
Next, because many counties receive no migrants at all we estimate a
linear probability model in which \(mig_{jt}\) takes on a value of one
if any migrants are observed moving to a particular county in a given
year. Finally, we calculate the share of New Orleans' area migration
that goes to each county.

The first two variables capture the intensive margin of migration, while
the third measures the extensive margin. These models describe the
distribution and magnitude of post-disaster migration. The final
specification speaks to the mix of migration across potential
destinations, holding constant the magnitude of migration flows.

\section{\texorpdfstring{Results
\label{sec:results}}{Results }}\label{results}

In Table \ref{tab:reg_main}, we present the results of a series of OLS
regressions related to the flow of migrants from the New Orleans area.
In each column, we present a particular transformation of the flow of
migrants from New Orleans to each destination county. Columns 1 and 2
correspond to the count of migrants and the inverse hyperbolic sine of
migrant count, Column 3 relates to a dummy variable for whether or not a
county had more than 10 migrants from any of the affected counties, and
Column 4 is each county's share of all migrants leaving the New Orleans
area.

In general, the results reflect relatively standard migration
preferences. Counties that are large, or close to the New Orleans area
are more likely to receive migrants than less populous counties or those
that are far from southern Louisiana. Similarly, counties with lower
unemployment and higher wages are more likely to receive migrants than
counties with less robust economies, although we find no statistically
significant effect of a destination county's median rent on migration
decisions. Unsurprisingly, in 2005, overall flows and the frequency with
which counties received migrants from the New Orleans area rose. While
the scale of the coefficients changes, the sign and statistical
significance of these impacts are consistent across each of our four
dependent variables.

In Table \ref{tab:reg2005}, we present estimation results with
interactions between our key explanatory variables and a dummy variable
for whether or not an observation corresponds with 2006. The interaction
term coefficients are interpreted as the effect of a given explanatory
variable on outflow migration in 2005 relative to the impact of that
variable in all other years. We attribute these differences to changes
in the migration decision-making process following a natural disaster.

In 2005, the relative importance of distance \emph{increased} relative
to the sample as a whole. While a one hundred-mile increase in distance
decreases the probability that a county receives migrants from the New
Orleans area by about 0.4 percentage points, in 2005 each additional one
hundred miles decreased the probability of migration by an additional
percentage point. Similarly, while each hundred miles of distance
reduced the number of migrants moving to a county by about one
indivudal, in 2005 each hundred miles of distances reduced the number of
migrants by about 8 individuals.

In general, economic factors were less important in determining
potential destinations than in non-disaster affected years. While
migrants tend towards areas with low levels of unemployment in most
years, the countervaling effect in 2005 was so large that migrants
actually moved towards counties with high unemployment rates. Similarly,
while there is not a statistically significant effect of average monthly
rent on migrant flows in most years, in the inverse hyperbolic sine and
linear probability models, 2005 migrants moved towards areas that had
high rents rather than low ones. On the other hand, in these models the
impact of average pay on migration became more important than most
years.

There is less evidence of a differential effect of pull factors in 2005
when we examine the share of migrants that moved to a particular county.
In that specification, the only statistically significant interaction
effect is on the unemployment rate, again signaling that economic
conditions were less important in 2005 than in other years. Across the
other explanatory variables, however, there are no statistically
significant effects. This may indicate that the results in the other
specifications are driven by \emph{new} migrants rather than shifts in
the preferences of existing migrants.

Together, these results raise concern regarding the ability of migration
to serve as a mechanism that will limit future disaster losses. The
results broadly indicate that in the aftermath of a disaster, migrants
focus primarily on proximity, to the detriment of economic
considerations. Because post-disaster migrants tend to move to areas
with weaker economies than non-disaster migrants, these migrants may
have difficulty finding new jobs and recovering from the financial
stress of the disaster. More importantly, under the assumption that
disaster exposure is spatially correlated, the preference for close
destinations suggests that migration is not an effective tool for
reducing future disaster losses. Migrants are merely shifting from one
risky county towards another risky county.

\section{\texorpdfstring{Conclusion
\label{sec:conclusion}}{Conclusion }}\label{conclusion}

This study has focused on the characteristics of the destinations of
post-Katrina migrants from the New Orleans area using IRS data on the
movement of tax returns and exemptions in order to explain the pattern
of out-migration. An understanding of the factors driving post-disaster
migration is important both in planning for shifts in population and in
assessing future damages from natural disasters.

In most years, migration away from the New Orleans area corresponds with
traditional gravity model results. Migrants prefer close destinations to
distant ones, and tend towards large, economically strong counties
rather than rural ones with fewer economic prospects.

In the immediate aftermath of Hurricane Katrina, however, the forces
driving migration decisions shifted. Proximity became even more
important to the migration destination decision than in most years. The
marginal impact of each one-hundred miles of distance from the New
Orleans area more than doubled in three of our four dependent variable
formations. In normal times, a county that was one-hundred miles further
from New Orleans would expect around 0.4\% fewer migrants but in 2005,
it would receive 1.4\% fewer migrants. At the same time, economic
considerations became less important than in non-disaster years. The
change in preferences is sufficiently strong to result in migrants
moving to areas with weak economies rather areas with strong ones.

Our results suggest the need for caution when projecting the benefits of
post-disaster migration as a tool for mitigating future disaster
damages. Disaster-migrants tend to relocate to destinations that are
close to their disaster-afflicted origin. Because disaster risks are
correlated spatially, this suggests that those impacted by disasters
will move to locations that are also susceptible to natural disasters
and that aggregate exposure to disaster risks will remain relatively
unchanged. Further, post-disaster migrants are less likely to recover
rapidly economically because they place little consideration on economic
conditions in the migration decision.

\newpage

\section{References}\label{references}

\hypertarget{refs}{}
\hypertarget{ref-black2011effect}{}
Black, Richard, W Neil Adger, Nigel W Arnell, Stefan Dercon, Andrew
Geddes, and David Thomas. 2011. ``The Effect of Environmental Change on
Human Migration.'' \emph{Global Environmental Change} 21. Elsevier:
S3--S11.

\hypertarget{ref-borjas1987self}{}
Borjas, George J. 1987. ``Self-Selection and the Earnings of
Immigrants.'' \emph{American Economic Review} 77 (4): 531--53.

\hypertarget{ref-chiswick1999immigrants}{}
Chiswick, Barry R. 1999. ``Are Immigrants Favorably Self-Selected?''
\emph{American Economic Review} 89 (2). JSTOR: 181--85.

\hypertarget{ref-elliott2006race}{}
Elliott, James R, and Jeremy Pais. 2006. ``Race, Class, and Hurricane
Katrina: Social Differences in Human Responses to Disaster.''
\emph{Social Science Research} 35 (2). Elsevier: 295--321.

\hypertarget{ref-fielding2011impacts}{}
Fielding, AJ. 2011. ``The Impacts of Environmental Change on Uk Internal
Migration.'' \emph{Global Environmental Change} 21. Elsevier:
S121--S130.

\hypertarget{ref-findlay2011migrant}{}
Findlay, Allan M. 2011. ``Migrant Destinations in an Era of
Environmental Change.'' \emph{Global Environmental Change} 21. Elsevier:
S50--S58.

\hypertarget{ref-fussell2014recovery}{}
Fussell, Elizabeth, Katherine J Curtis, and Jack DeWaard. 2014.
``Recovery Migration to the City of New Orleans After Hurricane Katrina:
A Migration Systems Approach.'' \emph{Population and Environment} 35
(3). Springer: 305--22.

\hypertarget{ref-gray2014studying}{}
Gray, Clark, Elizabeth Frankenberg, Thomas Gillespie, Cecep Sumantri,
and Duncan Thomas. 2014. ``Studying Displacement After a Disaster Using
Large-Scale Survey Methods: Sumatra After the 2004 Tsunami.''
\emph{Annals of the Association of American Geographers} 104 (3). Taylor
\& Francis: 594--612.

\hypertarget{ref-grada1997migration}{}
Gráda, Cormac Ó, and Kevin H O'Rourke. 1997. ``Migration as Disaster
Relief: Lessons from the Great Irish Famine.'' \emph{European Review of
Economic History} 1 (1). Oxford University Press: 3--25.

\hypertarget{ref-greenwood1975research}{}
Greenwood, Michael J. 1975. ``Research on Internal Migration in the
United States: A Survey.'' \emph{Journal of Economic Literature}. JSTOR,
397--433.

\hypertarget{ref-greenwood1985human}{}
---------. 1985. ``Human Migration: Theory, Models, and Empirical
Studies.'' \emph{Journal of Regional Science} 25 (4). Wiley Online
Library: 521--44.

\hypertarget{ref-groen2008hurricane}{}
Groen, Jeffrey A, and Anne E Polivka. 2008. ``Hurricane Katrina
Evacuees: Who They Are, Where They Are, and How They Are Faring.''
\emph{Monthly Labor Review} 131. HeinOnline: 32.

\hypertarget{ref-groen2010going}{}
---------. 2010. ``Going Home After Hurricane Katrina: Determinants of
Return Migration and Changes in Affected Areas.'' \emph{Demography} 47
(4). Springer: 821--44.

\hypertarget{ref-mcintosh2008measuring}{}
McIntosh, Molly Fifer. 2008. ``Measuring the Labor Market Impacts of
Hurricane Katrina Migration: Evidence from Houston, Texas.''
\emph{American Economic Review} 98 (2). JSTOR: 54--57.

\hypertarget{ref-mckenzie2010self}{}
McKenzie, David, and Hillel Rapoport. 2010. ``Self-Selection Patterns in
Mexico-Us Migration: The Role of Migration Networks.'' \emph{Review of
Economics and Statistics} 92 (4). MIT Press: 811--21.

\hypertarget{ref-nifo2014institutions}{}
Nifo, Annamaria, and Gaetano Vecchione. 2014. ``Do Institutions Play a
Role in Skilled Migration? The Case of Italy.'' \emph{Regional Studies}
48 (10). Taylor \& Francis: 1628--49.

\hypertarget{ref-rupasingha2015rural}{}
Rupasingha, Anil, Yongzheng Liu, and Mark Partridge. 2015. ``Rural
Bound: Determinants of Metro to Non-Metro Migration in the United
States.'' \emph{American Journal of Agricultural Economics} 97 (3).
Oxford University Press: 680--700.

\hypertarget{ref-weber2012displaced}{}
Weber, Lynn, and Lori A Peek. 2012. \emph{Displaced: Life in the Katrina
Diaspora}. University of Texas Press.

\hypertarget{ref-yun2016day}{}
Yun, Seong Do, and Brigitte S Waldorf. 2016. ``The Day After the
Disaster: Forced Migration and Income Loss After Hurricanes Katrina and
Rita.'' \emph{Journal of Regional Science} 56 (3). Wiley Online Library:
420--41.

\clearpage

\section{Tables and Figures}\label{tables-and-figures}

\begin{longtable}[]{@{}lrl@{}}
\caption{Most Common State Destination for Migrants in 2005
\label{tab:commondeststate}}\tabularnewline
\toprule
State & Migrants & Percentage of Total\tabularnewline
\midrule
\endfirsthead
\toprule
State & Migrants & Percentage of Total\tabularnewline
\midrule
\endhead
Texas & 73,252 & 40.3\%\tabularnewline
Louisiana & 45,014 & 24.8\%\tabularnewline
Georgia & 14,480 & 7.96\%\tabularnewline
Mississippi & 10,327 & 5.68\%\tabularnewline
Florida & 5,951 & 3.27\%\tabularnewline
Tennessee & 4,896 & 2.69\%\tabularnewline
\bottomrule
\end{longtable}

\clearpage

\begin{longtable}[]{@{}lllrl@{}}
\caption{Most Common County Destination for Migrants in 2005
\label{tab:commondest}}\tabularnewline
\toprule
FIPS & County & State & Migrants & Percentage of Total\tabularnewline
\midrule
\endfirsthead
\toprule
FIPS & County & State & Migrants & Percentage of Total\tabularnewline
\midrule
\endhead
48201 & Harris & Texas & 38,033 & 20.9\%\tabularnewline
22033 & East Baton Rouge & Louisiana & 14,291 & 7.86\%\tabularnewline
48113 & Dallas & Texas & 9,971 & 5.48\%\tabularnewline
48439 & Tarrant & Texas & 5,575 & 3.07\%\tabularnewline
22105 & Tangipahoa & Louisiana & 4,758 & 2.62\%\tabularnewline
22055 & Lafayette & Louisiana & 3,377 & 1.86\%\tabularnewline
48029 & Bexar & Texas & 3,114 & 1.71\%\tabularnewline
22005 & Ascension & Louisiana & 2,912 & 1.6\%\tabularnewline
13089 & Dekalb & Georgia & 2,783 & 1.53\%\tabularnewline
47157 & Shelby & Tennessee & 2,769 & 1.52\%\tabularnewline
\bottomrule
\end{longtable}

\clearpage

\begin{longtable}[]{@{}lllll@{}}
\caption{\label{tab:reg_main}Effect of Destination County
Characteristics on New Orleans Outflow}\tabularnewline
\toprule
term & Flow & IHS & LP & Share\tabularnewline
\midrule
\endfirsthead
\toprule
term & Flow & IHS & LP & Share\tabularnewline
\midrule
\endhead
Intercept & 14.99 & -0.0308 & -0.0197 & 0.0458*\tabularnewline
& (12.7717) & (0.1154) & (0.023) & (0.0278)\tabularnewline
Population (Millions) & 193.0825* & 2.2278*** & 0.3989*** &
0.4739**\tabularnewline
& (110.3735) & (0.5207) & (0.1027) & (0.2188)\tabularnewline
Distance (Hundreds of Miles) & -1.8106*** & -0.0268*** & -0.0048*** &
-0.005***\tabularnewline
& (0.4174) & (0.003) & (5e-04) & (0.0012)\tabularnewline
Unemployment Rate & -0.1306 & -0.0162*** & -0.0034*** &
-0.0015**\tabularnewline
& (0.2915) & (0.0035) & (7e-04) & (7e-04)\tabularnewline
Annual Pay (Thousands of USD) & 0.7592* & 0.0176*** & 0.0036*** &
0.002*\tabularnewline
& (0.4331) & (0.0047) & (9e-04) & (0.001)\tabularnewline
Median Monthly Rent & -3.9228 & 0.019 & 0.0056* & -0.0072\tabularnewline
& (3.7937) & (0.0155) & (0.0031) & (0.0076)\tabularnewline
Non-Metro & 4.1423 & -0.1107** & -0.0268*** & 0.0136\tabularnewline
& (7.3912) & (0.0499) & (0.0096) & (0.0168)\tabularnewline
Is 2005 & 51.6484*** & 0.5169*** & 0.092*** & -1e-04\tabularnewline
& (12.9228) & (0.025) & (0.005) & (0.0038)\tabularnewline
Adjusted R-Squared & 0.037 & 0.306 & 0.288 & 0.114\tabularnewline
Observations & 34,045 & 34,045 & 34,045 & 34,045\tabularnewline
\bottomrule
\end{longtable}

\newpage

\scriptsize

\begin{longtable}[]{@{}lllll@{}}
\caption{\label{tab:reg2005}Effect of Destination County Characteristics
on New Orleans Outflow - 2005 Interactions}\tabularnewline
\toprule
term & Flow & IHS & LP & Share\tabularnewline
\midrule
\endfirsthead
\toprule
term & Flow & IHS & LP & Share\tabularnewline
\midrule
\endhead
Intercept & 10.7085* & 0.0094 & -0.0115 & 0.0444*\tabularnewline
& (5.6626) & (0.1094) & (0.0225) & (0.025)\tabularnewline
Population (Millions) & 101.2034** & 2.1478*** & 0.3946*** &
0.4588**\tabularnewline
& (44.4983) & (0.4899) & (0.0996) & (0.1976)\tabularnewline
Distance (Hundreds of Miles) & -1.1058*** & -0.0226*** & -0.0041*** &
-0.005***\tabularnewline
& (0.261) & (0.003) & (5e-04) & (0.0012)\tabularnewline
Unemployment Rate & -0.5351*** & -0.0178*** & -0.0037*** &
-0.0019***\tabularnewline
& (0.1671) & (0.0033) & (7e-04) & (7e-04)\tabularnewline
Annual Pay (Thousands of USD) & 0.4672** & 0.016*** & 0.0034*** &
0.002**\tabularnewline
& (0.2207) & (0.0045) & (9e-04) & (0.001)\tabularnewline
Median Monthly Rent & -1.5468 & 0.0073 & 0.0031 & -0.0062\tabularnewline
& (1.5563) & (0.0149) & (0.0031) & (0.0068)\tabularnewline
Non-Metro & 2.9913 & -0.0598 & -0.0172* & 0.014\tabularnewline
& (3.6052) & (0.0482) & (0.0096) & (0.0162)\tabularnewline
Is 2005 & 79.8503 & -1.1087*** & -0.2173*** & 0.0052\tabularnewline
& (174.7202) & (0.2368) & (0.0418) & (0.0784)\tabularnewline
Population x 2005 & 1056.5554 & 0.6868** & 0.0062 &
0.1761\tabularnewline
& (777.6244) & (0.3442) & (0.0371) & (0.2543)\tabularnewline
Distance x 2005 & -7.4741*** & -0.0526*** & -0.0087*** &
3e-04\tabularnewline
& (1.719) & (0.0039) & (8e-04) & (6e-04)\tabularnewline
Unemployment Rate x 2005 & 10.1832** & 0.1105*** & 0.0198*** &
0.0072***\tabularnewline
& (3.995) & (0.0167) & (0.003) & (0.0025)\tabularnewline
Pay x 2005 & 2.9878 & 0.0263*** & 0.0043*** & -1e-04\tabularnewline
& (3.3237) & (0.0061) & (0.0011) & (0.0013)\tabularnewline
Monthly Rent x 2005 & -28.0041 & 0.2196*** & 0.0449*** &
-0.01\tabularnewline
& (32.9195) & (0.0313) & (0.0062) & (0.0125)\tabularnewline
Non-Metro x 2005 & 10.6583 & -0.4387*** & -0.083*** &
-0.0066\tabularnewline
& (33.1141) & (0.062) & (0.0129) & (0.0081)\tabularnewline
Adjusted R-Squared & 0.112 & 0.337 & 0.311 & 0.115\tabularnewline
Observations & 34,045 & 34,045 & 34,045 & 34,045\tabularnewline
\bottomrule
\end{longtable}


\end{document}
