\documentclass[]{article}
\usepackage{lmodern}
\usepackage{setspace}
\setstretch{2}
\usepackage{amssymb,amsmath}
\usepackage{ifxetex,ifluatex}
\usepackage{fixltx2e} % provides \textsubscript
\ifnum 0\ifxetex 1\fi\ifluatex 1\fi=0 % if pdftex
  \usepackage[T1]{fontenc}
  \usepackage[utf8]{inputenc}
\else % if luatex or xelatex
  \ifxetex
    \usepackage{mathspec}
  \else
    \usepackage{fontspec}
  \fi
  \defaultfontfeatures{Ligatures=TeX,Scale=MatchLowercase}
\fi
% use upquote if available, for straight quotes in verbatim environments
\IfFileExists{upquote.sty}{\usepackage{upquote}}{}
% use microtype if available
\IfFileExists{microtype.sty}{%
\usepackage{microtype}
\UseMicrotypeSet[protrusion]{basicmath} % disable protrusion for tt fonts
}{}
\usepackage[margin=1in]{geometry}
\usepackage{hyperref}
\hypersetup{unicode=true,
            pdftitle={The Effect of Disasters on Migration},
            pdfauthor={Jonathan Eyer, Adam Rose, Noah Miller; and Robert Dinterman},
            pdfborder={0 0 0},
            breaklinks=true}
\urlstyle{same}  % don't use monospace font for urls
\usepackage{longtable,booktabs}
\usepackage{graphicx,grffile}
\makeatletter
\def\maxwidth{\ifdim\Gin@nat@width>\linewidth\linewidth\else\Gin@nat@width\fi}
\def\maxheight{\ifdim\Gin@nat@height>\textheight\textheight\else\Gin@nat@height\fi}
\makeatother
% Scale images if necessary, so that they will not overflow the page
% margins by default, and it is still possible to overwrite the defaults
% using explicit options in \includegraphics[width, height, ...]{}
\setkeys{Gin}{width=\maxwidth,height=\maxheight,keepaspectratio}
\IfFileExists{parskip.sty}{%
\usepackage{parskip}
}{% else
\setlength{\parindent}{0pt}
\setlength{\parskip}{6pt plus 2pt minus 1pt}
}
\setlength{\emergencystretch}{3em}  % prevent overfull lines
\providecommand{\tightlist}{%
  \setlength{\itemsep}{0pt}\setlength{\parskip}{0pt}}
\setcounter{secnumdepth}{0}
% Redefines (sub)paragraphs to behave more like sections
\ifx\paragraph\undefined\else
\let\oldparagraph\paragraph
\renewcommand{\paragraph}[1]{\oldparagraph{#1}\mbox{}}
\fi
\ifx\subparagraph\undefined\else
\let\oldsubparagraph\subparagraph
\renewcommand{\subparagraph}[1]{\oldsubparagraph{#1}\mbox{}}
\fi

%%% Use protect on footnotes to avoid problems with footnotes in titles
\let\rmarkdownfootnote\footnote%
\def\footnote{\protect\rmarkdownfootnote}

%%% Change title format to be more compact
\usepackage{titling}

% Create subtitle command for use in maketitle
\newcommand{\subtitle}[1]{
  \posttitle{
    \begin{center}\large#1\end{center}
    }
}

\setlength{\droptitle}{-2em}
  \title{The Effect of Disasters on Migration}
  \pretitle{\vspace{\droptitle}\centering\huge}
  \posttitle{\par}
  \author{Jonathan Eyer, Adam Rose, Noah Miller\footnote{University of Southern
  California} \\ and Robert Dinterman\footnote{The Ohio State University}}
  \preauthor{\centering\large\emph}
  \postauthor{\par}
  \predate{\centering\large\emph}
  \postdate{\par}
  \date{04 June 2017}


\begin{document}
\maketitle
\begin{abstract}
While post-disaster migration can move vulnerable peoples from dangerous
regions to relatively safe areas, little is known about the processes
through which migrants select new homes. We refine a spatial econometric
model of migrant flows to examine the characteristics of the
destinations of migrants leaving the New Orleans area following
Hurricane Katrina in 2005 and in 2006 when the pressure to evacuate was
lessened. We find that migrants in 2005 settled close to New Orleans,
with little consideration for destination amenities or characteristics.
After the immediate threat had dissipated, however, migrants were more
likely to consider economic characteristics of destinations.
\end{abstract}

\newpage

\section{Introduction}\label{introduction}

Natural disasters can cause widespread destruction and weaken local
economies. These impacts can lead to permanent migration away from
disaster-affected areas. Such permanent, or even semi-permanent,
migration induced by natural disasters has the potential to
significantly reshape the distribution of national and global
populations and economies (see e.g. McIntosh (2008)). Moreover, because
migration moves people out of the path of some disasters, and
potentially into the path of other ones, post-disaster migration has
implications for the risks associated with future events (Gráda and
O'Rourke 1997). Finally, the migration itself and the loss of community
cohesion suggests the need for consideration of mental health support in
communities that will receive large numbers of disaster migrants (Weber
and Peek 2012).

Migration following large disasters is well-documented after major
events like Hurricane Katrina, the 2011 Tohoku earthquake/tsunami and
subsequent Fukushima nuclear disaster, and the 2004 Indian Ocean
tsunami. Because of the pressures placed on the affected population,
disasters can cause migration among a wider portion of the population
than those who migrate normally (in a non-disaster related context)
(Gray et al. 2014). While the propensity for disaster-affected
populations to migrate is documented, less is known about the
preferences that impact the destination of disaster-affiliated migrants.

Factors in migration decisions are generally framed in the context of
``push'' factors and ``pull'' factors. Push factors cause people to want
to leave the origin while pull factors cause people to want to go to a
specific destination. High unemployment in the origin might push people
to leave, for example. Similarly, a low cost of living might pull people
toward a particular destination. An understanding of these pull factors
is important for crafting natural disaster policies, understanding the
likely evolution of disaster damages, and evaluating the prospects for
repatriation. For example, if post-disaster migrants are credit
constrained and unable to move to the optimal location, government
subsidies for relocation costs might be justified. Similarly, if
post-disaster migrants move to other areas that are at high risk of
natural disasters, future disaster losses may actually increase
following the migration.

Hurricane Katrina, which struck New Orleans in 2005, provides an ideal
case study to examine the factors that influence the destination of
disaster migrants. Most residents of New Orleans evacuated prior to the
Hurricane, and following the storm most remaining residents were
evacuated by the Federal Emergency Management Agency (FEMA).
Approximately 1.5 million people evacuated the New Orleans area. 96\% of
New Orleans residents and 80\% of residents surrounding the city
eventually left their homes (Groen and Polivka 2008; Elliott and Pais
2006). While a large number of evacuees were relocated to Houston by
FEMA, Katrina evacuees relocated throughout the country. Nearly every
state received FEMA funding for costs associated with supporting
evacuees from Katrina. Many of those who evacuated following Hurricane
Katrina never returned to the New Orleans area. These permanent migrants
were generally younger, more likely to have children, and more likely to
be black than those who returned to New Orleans (Groen and Polivka
2010). There was also an increased flow of migrants from neighboring
communities in the years following Katrina compared to the years prior
to Katrina, indicating that those migrants who relocated to nearby
communities were more likely to return than those who relocated to
distant ones (Fussell, Curtis, and DeWaard 2014).

In this paper, we examine the migration pull factors in terms of
characteristics of the destinations of post-Katrina migration out of the
New Orleans area by using data on the movement of IRS return filings
between counties and a range of county-destination attributes. This
paper contributes to the literature by estimating the relative
importance of a range of factors in post-disaster relocation decisions.
This work conveys a range of policy implications surrounding disasters
and climate change. By identifying the characteristics that draw
migrants following natural disasters, we inform future migration
patterns as disasters grow more frequent. Our consideration of distance
in the relocation decision also highlights the extent to which
post-disaster migrants will be removed from similarly disaster-prone
areas. Finally, we contribute to a small but growing set of studies that
model migration in an explicitly spatial econometric context.

The rest of the paper proceeds as follows. In Section \ref{theory}, we
review the theoretical structure of migration decisions, in Section
\ref{data} discuss our data sources, in Section \ref{sec:meth} present
our estimating equations, and in Section \ref{sec:results} present our
results. \# Conceptual Underpinnings \label{theory}

From an economic standpoint, migration decisions are based on households
comparing their expected lifetime utility in their current location (the
origin) to a location to which they could move (the destination)
(Greenwood 1985; Greenwood 1975). Yun and Waldorf (2016) examine the
decision about whether or not to migrate in an expected lifetime utility
framework and focus on the extent to which Katrina induced migration by
those who would not otherwise have migrated. The utility that a
household expects to receive from living in a particular location
depends on economic variables such as the wages and cost-of-living
associated with an area, but also on non-economic variables such as
environmental amenities, family and social ties, and perceptions about
safety. A household will decide to migrate if the increase in expected
lifetime utility obtained by moving from the origin to the destination
exceeds the costs associated with moving. These costs include the
financial costs associated with moving, as well as more abstract factors
such as the social costs incurred by the move.

The decision to migrate is generally endogenous to migrant
characteristics. Highly-skilled migrants who expect to receive large
wage premiums are more likely to migrate than low-skilled workers, for
example Borjas (1987). Similarly, migration is costly, and those who are
less credit-constrained are more able to afford the upfront costs
associated with an optimal relocation decision (Chiswick 1999).

Natural disasters, however, cause exogenous variation in the expected
lifetime utility at the origin. For example, property damage would
require repair costs in order to stay at the origin, and a weakened
local economy would lower wages at the origin. Similarly, if a disaster
causes households to update their beliefs about the likelihood and
severity of subsequent events, this could lower the expected utility of
remaining in the origin. These effects would cause households to
re-evaluate their location decisions and potentially choose to migrate
due to the decreased expected life-time utility at their origin (Yun and
Waldorf 2016).

In the event of major natural disasters like Hurricane Katrina and the
Fukushima nuclear disaster, the push factors are relatively obvious;
people leave the origin because of mandatory evacuation requirements,
legal inability to return due to quarantines, loss of employment
opportunities, etc. It is less obvious what draws migrants to particular
locations following a disaster. One might be particularly concerned that
post-disaster migrants are systematically different than those who
choose to migrate under other circumstances. Disaster-related migrants,
for example, might feel compelled to relocate more quickly or have less
wealth with which to bear moving costs. Hence, they may not move to
optimal locations in comparison to normal circumstances, or what Yun and
Waldorf (2016) refer to as ``double-victimization.'' Black et al. (2011)
suggest that population movements due to disasters are typically short
distance, though this conclusion seems to be counter to what happened in
the aftermath of Hurricane Katrina.

Several variables have been suggested, and some tested, to explain the
pull factors. Many of these are traditional in the gravity model
literature of migration, such as wage and cost-of-living differentials,
distance, moving costs, and general economic health of the destination
(Borjas 1987; Rupasingha, Liu, and Partridge 2015). Broadening the
analysis leads to consideration of amenities, family ties, racial/ethnic
affinities, migration networks, and institutions (McKenzie and Rapoport
2010; Nifo and Vecchione 2014). The destination choice itself is
dependent on the reason that drives the individual to migrate (Findlay
2011). One might conclude that short or long-run hazard vulnerability
would be major considerations, but Black et al. (2011) and Fielding
(2011) emphasize the primacy of socioeconomic over environmental
variables in current migration decisions, though on the basis of only
anecdotal information.

\section{\texorpdfstring{Data \label{data}}{Data }}\label{data}

Our primary source of data is the Internal Revenue Service (IRS)
Statistics of Income Division's migration data. These data report the
flows of populations between counties based on changes in the location
from which tax returns are filed. The IRS reports both outflow migration
(tax returns and exemptions of filers who leave a county) as well as
inflow migration (tax returns and exemptions that enter a county). The
IRS data also include the number of filings and exemptions of people who
do not move, providing a base-level population value that is comparable
to the migration data. These reports include not only those filings that
change counties but also the number of claimed exemptions that change
counties and the annual gross adjusted income associated with the
filings. In order to ensure the privacy of individual filers, the IRS
suppresses observations in which fewer than ten filers migrated between
an origin-destination pair. We treat these values as true zeroes.

Given our focus on New Orleans, we restrict our interest to outflow
migration from the counties most severely affected by Hurricane Katrina
in 2005.{[}\^{}parish{]} We define our population affected by Hurricane
Katrina as those residing in Cameron Parish, Orleans Parish, Plaquemines
Parish, St.~Bernard Parish, and Jefferson Parish. While there was some
migration between affected regions (i.e., moving from a county that was
severely affected to one that was slightly less affected), we remove
these migrants from our sample to facilitate a simpler interpretation of
outflow migrants. {[}\^{}parish{]}: While Louisiana is organized into
parishes rather than counties, we will use the term counties throughout
this paper to facilitate discussion of destination locations.

We aggregate annual migration flows to each destination county across
the five highly-affected origin counties between 2000 and 2010 The
result is an 11-year panel of population flows to the 3,139
destination-county.\footnote{There are 3,144 counties and county
  equivalents in the U.S. and the five affected counties are removed
  from the set of potential destination counties.} There is a non-zero
number of migrants to approximately 5.4\% of the county-year
observations in our dataset. In 2005, however, over 13\% of US counties
received migrants from the affected area. Table
\textasciitilde{}\ref{desttable_year} shows the total number of migrants
from the affected area in each year of our sample, as well as the
percentage of destination counties that received migrants.

We supplement the IRS migration data with a number of explanatory
variables that might affect the relative attractiveness of a destination
county. Annual county-level unemployment rates are obtained from the
Bureau of Labor Statistics (BLS). Similarly, annual county-level racial
composition data is obtained from the U.S. Census' intra-decennial
Population Estimates Program. Average annual wage data is obtained the
the BLS' Quarterly Census of Wages. For each variable we merge these
data with the IRS migration data by county and year.

\section{\texorpdfstring{Methodology
\label{sec:meth}}{Methodology }}\label{methodology}

\subsection{OLS Models}\label{ols-models}

In order to understand how Hurricane Katrina affected migration
preferences, we estimate a series of models of migration outflow from
the five affected counties. We specify the model

\begin{equation} \label{eq:basereg}
    mig_{jt} = f(population_{jt}, g(distance_j), X_E, X_D, Time_t, Year 2005 \times population_{jt}, Year 2005 \times g(distance_j), Year 2005 \times X_E, Year 2005 \times X_D)
\end{equation}

where \(mig_{jt}\) is the number of total migrants (filers and
exemptions) that moved from the counties affected by Hurricane Katrina
to county \(j\) in year \(t\), and \(X_E\) and \(X_D\) are matrices of
economic and demographic explanatory variables, respectively. The
economic explanatory variable matrix, \(X_E\) is composed of
\(unemploy_{jt}\), the unemployment rate in county \(j\) in year \(t\)
and \(pay_{jt}\), the average annual wage rate in county \(j\) at year
\(t\). The demographic explanatory variables matrix, \(X_D\), is
composed of \(black_{jt}\), which is the percentage of county \(j's\)
population in year \(t\) that is black, and \(LA_j\), a dummy variable
that takes on a value of one if county \(j\) is in Louisiana and a value
of zero otherwise. Finally, \(g(distance_j)\) is a function of the
Euclidean distance between county \(j\) and Orleans Parish. We consider
four distance specifications as controls: linear, cubic, and quartic
functions of the Euclidean distance and a restricted cubic
spline.\footnote{The restricted cubic spline fits a series of cubic
  polynomials to the data. This allows different polynomial structures
  at different portions of the data's domain. The polynomials are
  constrained to ensure continuity across the polynomial specifications.}

Our key explanatory variable is Year 2005 which is a dummy variable that
takes on a value of one if an observation occurs in 2005 and a zero
otherwise. We interact this dummy variable with each of our other
explanatory variables, so that the interaction terms capture the change
in preferences over each pull factor variable relative to the other
years in our sample, when a major disaster did not strike. If Hurricane
Katrina shifted the relative importance of pull factors in the
destination-selection process, we would expect these interaction terms
to be statistically significant.

We consider a set of dependent variables for our regressions, and
estimate the model separately with each potential dependent variable.
First, we consider the number of migrants to county \(j\) itself. While
this is not a gravity model per se because we do not estimate logs of
migration, it remains in the spirit of a gravity model in which distance
and other amenities affect migrant decisions. Next, because many
counties receive no migrants at all we estimate a linear regression in
which \(mig_{jt}\) takes on a value of one if any migrants are observed
moving to a particular county in a given year. Finally, we estimate the
percentage of migrants from the affected five-counties to each
destination county and the percentage of incoming migrants in each
destination county that came from the Katrina-affected counties. The
former specification allows us to examine how Hurricane Katrina affected
the distribution of migrant destinations in a way that is invariant to
the total number of migrants. The latter informs the effect of
post-disaster migrants on destination counties in the sense that
counties will likely be able to absorb some number of migrants from a
disaster-afflicted region, but as the percentage of a county's new
population that is migrating away from a disaster rises there is
increasing strain placed on the destination county.

Next, we expand our consideration to outflow migration from the greater
New Orleans area in 2006. This allows us to consider migrants who did
not permanently or semi-permanently leave the area immediately in the
wake of Hurricane Katrina, but rather in subsequent years. Such migrants
are relatively more likely to have left the New Orleans area because of
a weakened economy than those who left in 2005.

We again estimate Equation \textasciitilde{}\ref{eq:basereg}, but we
include a dummy variable for whether or not an observation corresponds
to 2006 as well as the associated interaction terms. In the same spirit
as the 2005 interaction terms, these coefficients are interpreted as the
change in migration pull factors in the medium-term aftermath of
Hurricane Katrina. This allows us to examine the migratory behavior of
New Orleans area residents in 2006. Because these people weathered the
immediate brunt of the storm and then chose to leave, these effects are
likely to capture behavior that is more focused on leaving the weakened
city for better opportunities.

\section{\texorpdfstring{Results
\label{sec:results}}{Results }}\label{results}

\subsection{Baseline Results}\label{baseline-results}

In Table \textasciitilde{}\ref{tab:lpm}, we present the results of a
linear probability model of whether or not any migrants were observed
moving to a particular county from the New Orleans area. In general, the
results across the entire 2000-2010 time range reflect relatively
standard migration preferences. Counties that are large, or close to the
New Orleans area are more likely to receive migrants than less populous
counties or those that are far from southern Louisiana. Similarly,
counties with lower unemployment and higher wages are more likely to
receive migrants than counties with less robust economies. Counties with
large black populations are more alluring to migrants, consistent with
the largely black population of New Orleans moving to areas with which
they have social and family connections.

In 2005, the year of the disaster, the relative importance of distance
\emph{increased} relative to the sample as a whole. While a one
hundred-mile increase in distance decreases the probability that a
county receives migrants from the New Orleans area by about 0.3
percentage points, in 2005 each additional one hundred miles decreased
the probability of migration by an additional 0.5 percentage points. In
general, economic factors were less important in determining potential
destinations than in non-disaster affected years, the effect of the
unemployment rate on migrant destinations in 2005 is statistically
indistinguishable from zero. One exception to the economic effect is
through the channel of wages, which became more alluring when the
disaster struck. Counties with a greater percentage of black residents
were more likely to receive migrants in 2005 than in other years,
further evidence of the reliance on community ties when disasters
strike.

In 2006, we find greater emphasis on economic characteristics than in
the sample as a whole. Counties with low unemployment rates and a low
cost of living are more likely to receive migrant flows from the Katrina
area. We also find a slight reduction in the importance of population in
determining the destination to which migrants move.

Table \textasciitilde{}\ref{reg:fullsample} presents the results for the
OLS models based on migration out of the New Orleans area in 2005. In
each column we present an alternative specification for the functional
form by which distance is related to the number of migrants who move to
a given county. Across each model, we find consistent evidence that
distance between a county and the New Orleans area and the population of
the destination county are statistically significant indicators of the
number of migrants who move to a particular county. Louisiana counties
generally receive a greater number of migrants than outside counties
across each of the specifications. \%Is this Table included?? I don't
see it. The following 4 paragraphs all seem to discuss this missing
Table, so I can't double check their validity as of this moment.

Because it is difficult to conceptualize the marginal effect of distance
on migrants across each of our specifications, we present a graph of
each distance control in Figure \textasciitilde{}\ref{fig:dist2005}.
Across each of our distance control specifications, the number of
migrants moving to a county is declining in distance.\footnote{Population
  flows are relatively sparse between counties more than 1000 miles
  apart, and after 1000 miles the functional forms of the destination
  polynomials force the polynomials either upward or downward.} These
lines can be interpreted by comparing the functional value between
various points along the x-axis. For example, the value of the spline
function in green is 0 at a distance of 0 and approximately -1200 at a
distance of 1000 miles. This means that a county 1000 miles from New
Orleans would receive 1200 fewer migrants than a hypothetical,
unevacuated county that was 0 miles from New Orleans. In each
specification except the linear, the effect of distance is declining
over the relevant range. The effect of a marginal mile of distance
begins to flatten around 500 miles from the New Orleans area.

Migrants generally show a preference for more populous counties than for
less populous counties. Across each distance specification the number of
migrants who move to a given county increases by approximately 1,150 for
every million residents of the destination county. For comparison, note
that the effect of a marginal one million residents in a county has
approximately the same draw as a county that is 1000 miles closer to New
Orleans than an alternative.

We also find that migrants are much more likely to move to a county that
is in Louisiana than one that is outside of Louisiana. This effect is
relatively consistent across each of our distance controls so it is
unlikely that this is merely capturing preferences for close counties
over distant ones. Rather it is likely that residents of New Orleans
feel an affiliation with the state of Louisiana and prefer to stay in
the state for these reasons. This is consistent with the anchoring
literature, which suggests that people feel a special affinity with the
place that they perceive as ``home'' (Chamlee-wright and Storr 2009). A
county in Louisiana is expected to receive about 1,000 more migrants
than a comparable county not in the state. This effect is roughly
equivalent to the number of migrants associated with an additional
million residents in the destination county.

\section{\texorpdfstring{Conclusion
\label{sec:conclusion}}{Conclusion }}\label{conclusion}

This study has focused on the characteristics of the destinations of
post-Katrina migrants from the New Orleans area using IRS data on the
movement of tax returns and exemptions in order to explain the pattern
of out-migration. An understanding of the factors driving post-disaster
migration is important both in planning for shifts in population and in
assessing future damages from natural disasters.

We find that in 2005 -- in the immediate aftermath of Hurricane Katrina
-- that distance and destination counties' percentage black population
were the primary drivers of relocation decisions. Households also opted
to live in Louisiana at a greater rate than one would expect based on
distances alone. While it may appear surprising that unemployment did
not provide statistically significant effects on migration, it is
unsurprising given the urgency with which people left New Orleans in the
immediate aftermath of the Katrina and the fact that FEMA determined at
least the initial destination of many of them. Indeed, after removing
Harris County -- which received a relatively high proportion of the
forced migrants from Katrina -- from the sample, unemployment and wage
rates appear to have an effect on migrant destinations, although the
effect remains small relative to destination population and distance.

When we focus on those who left New Orleans in 2006, however, we find
explanatory power behind the traditional covariates suggested in gravity
models of migration flows. In particular, we find that people are moving
to areas with relatively higher annual wages and relatively lower costs
of living. As in 2005, we find a preference for moving to larger and
closer counties. Because New Orleans was still severely weakened in
2006, these results can be interpreted as those who remained in New
Orleans in the year following Katrina moving away from a weakened city
the following year.

In each case, these results are robust to potential misspecification in
the spatial dependence in both migration flows and the error term of the
estimating equation. Failure to account for spatial correlation can bias
OLS estimates. The consistency of our regression results between the
spatial and OLS specifications, however, should provide some measure of
confidence in other county-level migration models that do not explicitly
model spatial correlation.

Our results suggest the need for caution when projecting the benefits of
post-disaster migration as a tool for mitigating future disaster
damages. Disaster-migrants tend to relocate to destinations that are
close to their disaster-afflicted origin. Because disaster risks are
correlated spatially, this suggests that those impacted by disasters
will move to locations that are also susceptible to natural disasters
and that aggregate exposure to disaster risks will remain relatively
unchanged. Still, those who leave the disaster-afflicted area tend to
move to high-population, urban areas rather than rural regions. If
disasters that strike rural areas similarly induce migration towards
population centers, such that migrants benefit from the greater economic
opportunity in the cities.

\newpage

\section{References}\label{references}

\hypertarget{refs}{}
\hypertarget{ref-black2011effect}{}
Black, Richard, W Neil Adger, Nigel W Arnell, Stefan Dercon, Andrew
Geddes, and David Thomas. 2011. ``The Effect of Environmental Change on
Human Migration.'' \emph{Global Environmental Change} 21. Elsevier:
S3--S11.

\hypertarget{ref-borjas1987self}{}
Borjas, George J. 1987. ``Self-Selection and the Earnings of
Immigrants.'' \emph{American Economic Review} 77 (4): 531--53.

\hypertarget{ref-chamlee2009there}{}
Chamlee-wright, Emily, and Virgil Henry Storr. 2009. ```There's No Place
Like New Orleans': Sense of Place and Community Recovery in the Ninth
Ward After Hurricane Katrina.'' \emph{Journal of Urban Affairs} 31 (5).
Wiley Online Library: 615--34.

\hypertarget{ref-chiswick1999immigrants}{}
Chiswick, Barry R. 1999. ``Are Immigrants Favorably Self-Selected?''
\emph{American Economic Review} 89 (2). JSTOR: 181--85.

\hypertarget{ref-elliott2006race}{}
Elliott, James R, and Jeremy Pais. 2006. ``Race, Class, and Hurricane
Katrina: Social Differences in Human Responses to Disaster.''
\emph{Social Science Research} 35 (2). Elsevier: 295--321.

\hypertarget{ref-fielding2011impacts}{}
Fielding, AJ. 2011. ``The Impacts of Environmental Change on Uk Internal
Migration.'' \emph{Global Environmental Change} 21. Elsevier:
S121--S130.

\hypertarget{ref-findlay2011migrant}{}
Findlay, Allan M. 2011. ``Migrant Destinations in an Era of
Environmental Change.'' \emph{Global Environmental Change} 21. Elsevier:
S50--S58.

\hypertarget{ref-fussell2014recovery}{}
Fussell, Elizabeth, Katherine J Curtis, and Jack DeWaard. 2014.
``Recovery Migration to the City of New Orleans After Hurricane Katrina:
A Migration Systems Approach.'' \emph{Population and Environment} 35
(3). Springer: 305--22.

\hypertarget{ref-gray2014studying}{}
Gray, Clark, Elizabeth Frankenberg, Thomas Gillespie, Cecep Sumantri,
and Duncan Thomas. 2014. ``Studying Displacement After a Disaster Using
Large-Scale Survey Methods: Sumatra After the 2004 Tsunami.''
\emph{Annals of the Association of American Geographers} 104 (3). Taylor
\& Francis: 594--612.

\hypertarget{ref-grada1997migration}{}
Gráda, Cormac Ó, and Kevin H O'Rourke. 1997. ``Migration as Disaster
Relief: Lessons from the Great Irish Famine.'' \emph{European Review of
Economic History} 1 (1). Oxford University Press: 3--25.

\hypertarget{ref-greenwood1975research}{}
Greenwood, Michael J. 1975. ``Research on Internal Migration in the
United States: A Survey.'' \emph{Journal of Economic Literature}. JSTOR,
397--433.

\hypertarget{ref-greenwood1985human}{}
---------. 1985. ``Human Migration: Theory, Models, and Empirical
Studies.'' \emph{Journal of Regional Science} 25 (4). Wiley Online
Library: 521--44.

\hypertarget{ref-groen2008hurricane}{}
Groen, Jeffrey A, and Anne E Polivka. 2008. ``Hurricane Katrina
Evacuees: Who They Are, Where They Are, and How They Are Faring.''
\emph{Monthly Labor Review} 131. HeinOnline: 32.

\hypertarget{ref-groen2010going}{}
---------. 2010. ``Going Home After Hurricane Katrina: Determinants of
Return Migration and Changes in Affected Areas.'' \emph{Demography} 47
(4). Springer: 821--44.

\hypertarget{ref-mcintosh2008measuring}{}
McIntosh, Molly Fifer. 2008. ``Measuring the Labor Market Impacts of
Hurricane Katrina Migration: Evidence from Houston, Texas.''
\emph{American Economic Review} 98 (2). JSTOR: 54--57.

\hypertarget{ref-mckenzie2010self}{}
McKenzie, David, and Hillel Rapoport. 2010. ``Self-Selection Patterns in
Mexico-Us Migration: The Role of Migration Networks.'' \emph{Review of
Economics and Statistics} 92 (4). MIT Press: 811--21.

\hypertarget{ref-nifo2014institutions}{}
Nifo, Annamaria, and Gaetano Vecchione. 2014. ``Do Institutions Play a
Role in Skilled Migration? The Case of Italy.'' \emph{Regional Studies}
48 (10). Taylor \& Francis: 1628--49.

\hypertarget{ref-rupasingha2015rural}{}
Rupasingha, Anil, Yongzheng Liu, and Mark Partridge. 2015. ``Rural
Bound: Determinants of Metro to Non-Metro Migration in the United
States.'' \emph{American Journal of Agricultural Economics} 97 (3).
Oxford University Press: 680--700.

\hypertarget{ref-weber2012displaced}{}
Weber, Lynn, and Lori A Peek. 2012. \emph{Displaced: Life in the Katrina
Diaspora}. University of Texas Press.

\hypertarget{ref-yun2016day}{}
Yun, Seong Do, and Brigitte S Waldorf. 2016. ``The Day After the
Disaster: Forced Migration and Income Loss After Hurricanes Katrina and
Rita.'' \emph{Journal of Regional Science} 56 (3). Wiley Online Library:
420--41.

\clearpage

\section{Tables and Figures}\label{tables-and-figures}

\begin{longtable}[]{@{}lrl@{}}
\caption{Most Common State Destination for Migrants in 2005
\label{tab:commondeststate}}\tabularnewline
\toprule
State & Migrants & Percentage of Total\tabularnewline
\midrule
\endfirsthead
\toprule
State & Migrants & Percentage of Total\tabularnewline
\midrule
\endhead
Texas & 73,252 & 40.3\%\tabularnewline
Louisiana & 45,014 & 24.8\%\tabularnewline
Georgia & 14,480 & 7.96\%\tabularnewline
Mississippi & 10,327 & 5.68\%\tabularnewline
Florida & 5,951 & 3.27\%\tabularnewline
Tennessee & 4,896 & 2.69\%\tabularnewline
\bottomrule
\end{longtable}

\clearpage

\begin{longtable}[]{@{}lllrl@{}}
\caption{Most Common County Destination for Migrants in 2005
\label{tab:commondest}}\tabularnewline
\toprule
FIPS & County & State & Migrants & Percentage of Total\tabularnewline
\midrule
\endfirsthead
\toprule
FIPS & County & State & Migrants & Percentage of Total\tabularnewline
\midrule
\endhead
48201 & Harris & Texas & 38,033 & 20.9\%\tabularnewline
22033 & East Baton Rouge & Louisiana & 14,291 & 7.86\%\tabularnewline
48113 & Dallas & Texas & 9,971 & 5.48\%\tabularnewline
48439 & Tarrant & Texas & 5,575 & 3.07\%\tabularnewline
22105 & Tangipahoa & Louisiana & 4,758 & 2.62\%\tabularnewline
22055 & Lafayette & Louisiana & 3,377 & 1.86\%\tabularnewline
48029 & Bexar & Texas & 3,114 & 1.71\%\tabularnewline
22005 & Ascension & Louisiana & 2,912 & 1.6\%\tabularnewline
13089 & Dekalb & Georgia & 2,783 & 1.53\%\tabularnewline
47157 & Shelby & Tennessee & 2,769 & 1.52\%\tabularnewline
\bottomrule
\end{longtable}

\clearpage

\begin{longtable}[]{@{}lllll@{}}
\toprule
term & Flow & IHS & LP & Share\tabularnewline
\midrule
\endhead
(Intercept) & 14.8969 & -0.0325 & -0.02 & 0.0455\tabularnewline
& (12.7597) & (0.1154) & (0.023) & (0.0278)\tabularnewline
population & 193.1327* & 2.2289*** & 0.3992*** & 0.474**\tabularnewline
& (110.3825) & (0.521) & (0.1027) & (0.2188)\tabularnewline
distance & -1.8062*** & -0.0268*** & -0.0048*** &
-0.005***\tabularnewline
& (0.4165) & (0.003) & (5e-04) & (0.0012)\tabularnewline
un\_rate & -12.6374 & -1.605*** & -0.3361*** & -0.1494**\tabularnewline
& (29.0974) & (0.346) & (0.0689) & (0.0688)\tabularnewline
pay & 0.7527* & 0.0175*** & 0.0036*** & 0.002*\tabularnewline
& (0.4318) & (0.0047) & (9e-04) & (0.001)\tabularnewline
fmr & -3.9009 & 0.0195 & 0.0057* & -0.0071\tabularnewline
& (3.7921) & (0.0155) & (0.0031) & (0.0076)\tabularnewline
metro03nonmetro & 4.2191 & -0.1092** & -0.0266*** &
0.0138\tabularnewline
& (7.3925) & (0.0498) & (0.0096) & (0.0168)\tabularnewline
katrina & 51.6205*** & 0.5167*** & 0.092*** & -1e-04\tabularnewline
& (12.9152) & (0.025) & (0.005) & (0.0038)\tabularnewline
Adjusted R-Squared & 0.037 & 0.305 & 0.288 & 0.114\tabularnewline
\bottomrule
\end{longtable}

\clearpage

\begin{table}[!htbp]  \scriptsize \centering 
  \caption{Any Migration to a County - Linear Probability Model} 
  \label{tab:lpm} 
\begin{tabular}{@{\extracolsep{5pt}}lccc} 
\\[-1.8ex]\hline 
\hline \\[-1.8ex] 
\hline \\[-1.8ex] 
\hline \\[-1.8ex] 
 Population (Millions) & 0.264$^{***}$ & 0.261$^{***}$ & 0.261$^{***}$ \\ 
  & (0.071) & (0.069) & (0.069) \\ 
  & & & \\ 
 Distance (Hundreds of Miles) & $-$0.003$^{***}$ & $-$0.003$^{***}$ & $-$0.003$^{***}$ \\ 
  & (0.001) & (0.001) & (0.001) \\ 
  & & & \\ 
 Louisiana Dummy & 0.365$^{***}$ & 0.343$^{***}$ & 0.334$^{***}$ \\ 
  & (0.051) & (0.055) & (0.055) \\ 
  & & & \\ 
 Percentage Black & 0.161$^{***}$ & 0.136$^{***}$ & 0.133$^{***}$ \\ 
  & (0.026) & (0.027) & (0.027) \\ 
  & & & \\ 
 Unemployment Rate & $-$0.005$^{***}$ & $-$0.005$^{***}$ & $-$0.004$^{***}$ \\ 
  & (0.001) & (0.001) & (0.001) \\ 
  & & & \\ 
 Annual Average Pay & 0.005$^{***}$ & 0.004$^{***}$ & 0.004$^{***}$ \\ 
  & (0.001) & (0.001) & (0.001) \\ 
  & & & \\ 
 $Population (Millions) \times 2005$ &  & 0.032 & 0.066$^{*}$ \\ 
  &  & (0.030) & (0.040) \\ 
  & & & \\ 
 Distance (Hundreds of Miles) x 2005 &  & $-$0.005$^{***}$ & $-$0.005$^{***}$ \\ 
  &  & (0.001) & (0.001) \\ 
  & & & \\ 
 Louisiana Dummy x 2005 &  & 0.301$^{***}$ & 0.311$^{***}$ \\ 
  &  & (0.062) & (0.062) \\ 
  & & & \\ 
 Percentage Black x 2005 &  & 0.284$^{***}$ & 0.311$^{***}$ \\ 
  &  & (0.046) & (0.047) \\ 
  & & & \\ 
 Unemployment Rate x 2005 &  & $-$0.001 & $-$0.005$^{**}$ \\ 
  &  & (0.002) & (0.003) \\ 
  & & & \\ 
 Annual Average Pay x 2005 &  & 0.006$^{***}$ & 0.009$^{***}$ \\ 
  &  & (0.001) & (0.001) \\ 
  & & & \\ 
 $Population (Millions) x 2006$ &  &  & $-$0.033$^{*}$ \\ 
  &  &  & (0.019) \\ 
  & & & \\ 
 Distance (Hundreds of Miles) x 2006 &  &  & $-$0.001$^{***}$ \\ 
  &  &  & (0.0004) \\ 
  & & & \\ 
 Louisiana Dummy x 2006 &  &  & 0.108$^{**}$ \\ 
  &  &  & (0.045) \\ 
  & & & \\ 
 Percentage Black x 2006 &  &  & 0.013 \\ 
  &  &  & (0.021) \\ 
  & & & \\ 
 Unemployment Rate x 2006 &  &  & $-$0.001 \\ 
  &  &  & (0.001) \\ 
  & & & \\ 
 Annual Average Pay x 2006 &  &  & 0.0004 \\ 
  &  &  & (0.001) \\ 
  & & & \\ 
 Constant & $-$0.025 & $-$0.021 & $-$0.006 \\ 
  & (0.020) & (0.020) & (0.020) \\ 
  & & & \\ 
\hline \\[-1.8ex] 
Observations & 34,359 & 34,359 & 34,359 \\ 
R$^{2}$ & 0.353 & 0.377 & 0.372 \\ 
Adjusted R$^{2}$ & 0.352 & 0.376 & 0.372 \\  
\textit{Note:}  & \multicolumn{3}{r}{$^{*}$p$<$0.1; $^{**}$p$<$0.05; $^{***}$p$<$0.01} \\ 
\end{tabular} 
\end{table}

\clearpage

\begin{table}[!htbp] \scriptsize \centering 
  \caption{Migration to Each County as Share of Total New Orleans Out-Migration} 
  \label{tab:percmigols} 
\begin{tabular}{@{\extracolsep{5pt}}lccc} 
\hline \\[-1.8ex] 
  Population (Millions) & 0.003$^{**}$ & 0.002$^{**}$ & 0.002$^{***}$ \\ 
  & (0.001) & (0.001) & (0.001) \\ 
  & & & \\ 
 Distance (Hundreds of Miles) & $-$0.0001$^{***}$ & $-$0.00005$^{***}$ & $-$0.00005$^{***}$ \\ 
  & (0.00002) & (0.00002) & (0.00002) \\ 
  & & & \\ 
 Louisiana Dummy & 0.007$^{**}$ & 0.007$^{**}$ & 0.008$^{**}$ \\ 
  & (0.003) & (0.003) & (0.003) \\ 
  & & & \\ 
 Percentage Black & $-$0.0005 & $-$0.001 & $-$0.001 \\ 
  & (0.001) & (0.001) & (0.001) \\ 
  & & & \\ 
 Unemployment Rate & $-$0.00003$^{***}$ & $-$0.00003$^{***}$ & $-$0.00003$^{***}$ \\ 
  & (0.00001) & (0.00001) & (0.00001) \\ 
  & & & \\ 
 Annual Average Pay & 0.00001 & 0.00001 & 0.00001 \\ 
  & (0.00001) & (0.00001) & (0.00001) \\ 
  & & & \\ 
 $Population (Millions) x 2005$ &  & 0.001 & 0.001 \\ 
  &  & (0.002) & (0.002) \\ 
  & & & \\ 
 Distance (Hundreds of Miles) x 2005 &  & $-$0.00002 & $-$0.00002 \\ 
  &  & (0.00002) & (0.00002) \\ 
  & & & \\ 
 Louisiana Dummy x 2005 &  & $-$0.004 & $-$0.004 \\ 
  &  & (0.003) & (0.003) \\ 
  & & & \\ 
 Percentage Black x 2005 &  & 0.001 & 0.001 \\ 
  &  & (0.001) & (0.001) \\ 
  & & & \\ 
 Unemployment Rate x 2005 &  & 0.00005$^{*}$ & 0.00005$^{**}$ \\ 
  &  & (0.00002) & (0.00003) \\ 
  & & & \\ 
 Annual Average Pay x 2005 &  & $-$0.00000 & $-$0.00000 \\ 
  &  & (0.00002) & (0.00002) \\ 
  & & & \\ 
 $Population (Millions) x 2006$ &  &  & 0.001 \\ 
  &  &  & (0.001) \\ 
  & & & \\ 
 Distance (Hundreds of Miles) x 2006 &  &  & $-$0.00001 \\ 
  &  &  & (0.00001) \\ 
  & & & \\ 
 Louisiana Dummy x 2006 &  &  & $-$0.003 \\ 
  &  &  & (0.003) \\ 
  & & & \\ 
 Percentage Black x 2006 &  &  & 0.0005 \\ 
  &  &  & (0.001) \\ 
  & & & \\ 
 Unemployment Rate x 2006 &  &  & 0.00003$^{**}$ \\ 
  &  &  & (0.00001) \\ 
  & & & \\ 
 Annual Average Pay x 2006 &  &  & 0.00000 \\ 
  &  &  & (0.00001) \\ 
  & & & \\ 
 Constant & 0.001 & 0.001 & 0.001 \\ 
  & (0.001) & (0.001) & (0.001) \\ 
  & & & \\ 
\hline \\[-1.8ex] 
Observations & 34,359 & 34,359 & 34,359 \\ 
R$^{2}$ & 0.117 & 0.120 & 0.121 \\ 
Adjusted R$^{2}$ & 0.117 & 0.119 & 0.120 \\
\hline \\[-1.8ex] 
\textit{Note:}  & \multicolumn{3}{r}{$^{*}$p$<$0.1; $^{**}$p$<$0.05; $^{***}$p$<$0.01} \\ 
\end{tabular} 
\end{table}

\clearpage

\begin{table}[!htbp] \scriptsize \centering 
  \caption{Migration from New Orleans Area as Share of Total Destination In-Migration} 
  \label{} 
\begin{tabular}{@{\extracolsep{5pt}}lccc} 
\\[-1.8ex]\hline 
\\[-1.8ex] & \multicolumn{3}{c}{percLA2fips} \\ 
\\[-1.8ex] & (1) & (2) & (3)\\ 
\hline \\[-1.8ex] 
 Population (Millions) & $-$0.0001 & $-$0.001 & $-$0.001 \\ 
  & (0.001) & (0.001) & (0.001) \\ 
  & & & \\ 
 Distance (Hundreds of Miles) & $-$0.0002$^{***}$ & $-$0.0002$^{***}$ & $-$0.0002$^{***}$ \\ 
  & (0.0001) & (0.0001) & (0.0001) \\ 
  & & & \\ 
 Louisiana Dummy & 0.060$^{***}$ & 0.047$^{***}$ & 0.048$^{***}$ \\ 
  & (0.013) & (0.014) & (0.015) \\ 
  & & & \\ 
 Percentage Black & 0.004 & $-$0.001 & $-$0.001 \\ 
  & (0.003) & (0.003) & (0.003) \\ 
  & & & \\ 
 Unemployment Rate & 0.00001 & $-$0.0001$^{***}$ & $-$0.0001$^{***}$ \\ 
  & (0.00004) & (0.00004) & (0.00004) \\ 
  & & & \\ 
 Annual Average Pay & 0.00000 & 0.00000 & 0.00000 \\ 
  & (0.00000) & (0.00000) & (0.00000) \\ 
  & & & \\ 
 $Population (Millions) x 2005$ & 0.0001 & 0.0001 & 0.0001 \\ 
  & (0.0001) & (0.0001) & (0.0001) \\ 
  & & & \\ 
 Distance (Hundreds of Miles) x 2005 &  & 0.006 & 0.006 \\ 
  &  & (0.004) & (0.004) \\ 
  & & & \\ 
 Louisiana Dummy x 2005 &  & $-$0.001$^{***}$ & $-$0.001$^{***}$ \\ 
  &  & (0.0001) & (0.0001) \\ 
  & & & \\ 
 Percentage Black x 2005 &  & 0.153$^{***}$ & 0.152$^{***}$ \\ 
  &  & (0.022) & (0.023) \\ 
  & & & \\ 
 Unemployment Rate x 2005 &  & 0.040$^{***}$ & 0.040$^{***}$ \\ 
  &  & (0.011) & (0.011) \\ 
  & & & \\ 
 Annual Average Pay x 2005 &  & 0.003$^{***}$ & 0.003$^{***}$ \\ 
  &  & (0.001) & (0.001) \\ 
  & & & \\ 
 $Population (Millions) x 2006$ &  & 0.0002 & 0.0002 \\ 
  &  & (0.0001) & (0.0001) \\ 
  & & & \\ 
 Distance (Hundreds of Miles) x 2006 &  &  & 0.002 \\ 
  &  &  & (0.001) \\ 
  & & & \\ 
 Louisiana Dummy x 2006 &  &  & $-$0.00000 \\ 
  &  &  & (0.00005) \\ 
  & & & \\ 
 Percentage Black x 2006 &  &  & $-$0.010 \\ 
  &  &  & (0.014) \\ 
  & & & \\ 
 Unemployment Rate x 2006 &  &  & 0.004 \\ 
  &  &  & (0.003) \\ 
  & & & \\ 
 Annual Average Pay x 2006 &  &  & 0.0003$^{***}$ \\ 
  &  &  & (0.0001) \\ 
  & & & \\ 
 pay06 &  &  & $-$0.00001 \\ 
  &  &  & (0.0001) \\ 
  & & & \\ 
 Constant & 0.001 & 0.001 & 0.002 \\ 
  & (0.002) & (0.002) & (0.002) \\ 
  & & & \\ 
\hline \\[-1.8ex] 
Observations & 32,679 & 32,679 & 32,679 \\ 
R$^{2}$ & 0.146 & 0.253 & 0.253 \\ 
Adjusted R$^{2}$ & 0.145 & 0.252 & 0.252 \\ 
\hline 
\hline \\[-1.8ex] 
\textit{Note:}  & \multicolumn{3}{r}{$^{*}$p$<$0.1; $^{**}$p$<$0.05; $^{***}$p$<$0.01} \\ 
\end{tabular} 
\end{table}

\begin{table}[!htbp] \scriptsize \centering 
  \caption{Inverse Hyperbolic Sine of Total Migrant Flow from New Orleans to Each County} 
  \label{} 
\begin{tabular}{@{\extracolsep{5pt}}lccc} 
\\[-1.8ex]\hline 
\cline{2-4} 

\hline \\[-1.8ex] 
  Population (Millions) & 1.411$^{***}$ & 1.361$^{***}$ & 1.354$^{***}$ \\ 
  & (0.343) & (0.323) & (0.321) \\ 
  & & & \\ 
 Distance (Hundreds of Miles) & $-$0.020$^{***}$ & $-$0.018$^{***}$ & $-$0.017$^{***}$ \\ 
  & (0.004) & (0.004) & (0.004) \\ 
  & & & \\ 
 Louisiana Dummy & 2.072$^{***}$ & 1.904$^{***}$ & 1.864$^{***}$ \\ 
  & (0.309) & (0.319) & (0.323) \\ 
  & & & \\ 
 Percentage Black & 0.686$^{***}$ & 0.547$^{***}$ & 0.531$^{***}$ \\ 
  & (0.131) & (0.129) & (0.128) \\ 
  & & & \\ 
 Unemployment Rate & $-$0.021$^{***}$ & $-$0.022$^{***}$ & $-$0.020$^{***}$ \\ 
  & (0.004) & (0.003) & (0.003) \\ 
  & & & \\ 
 Annual Average Pay & 0.022$^{***}$ & 0.019$^{***}$ & 0.017$^{***}$ \\ 
  & (0.005) & (0.005) & (0.005) \\ 
  & & & \\ 
 $Population (Millions) x 2005$ &  & 0.581$^{**}$ & 0.757$^{**}$ \\ 
  &  & (0.254) & (0.302) \\ 
  & & & \\ 
 Distance (Hundreds of Miles) x 2005 &  & $-$0.031$^{***}$ & $-$0.035$^{***}$ \\ 
  &  & (0.004) & (0.005) \\ 
  & & & \\ 
 Louisiana Dummy x 2005 &  & 2.222$^{***}$ & 2.262$^{***}$ \\ 
  &  & (0.272) & (0.280) \\ 
  & & & \\ 
 Percentage Black x 2005 &  & 1.575$^{***}$ & 1.702$^{***}$ \\ 
  &  & (0.211) & (0.222) \\ 
  & & & \\ 
 Unemployment Rate x 2005 &  & $-$0.005 & $-$0.029$^{**}$ \\ 
  &  & (0.012) & (0.013) \\ 
  & & & \\ 
 Annual Average Pay x 2005 &  & 0.033$^{***}$ & 0.050$^{***}$ \\ 
  &  & (0.006) & (0.007) \\ 
  & & & \\ 
 $Population (Millions) x 2006$ &  &  & $-$0.120$^{*}$ \\ 
  &  &  & (0.068) \\ 
  & & & \\ 
 Distance (Hundreds of Miles) x 2006 &  &  & $-$0.008$^{***}$ \\ 
  &  &  & (0.002) \\ 
  & & & \\ 
 Louisiana Dummy x 2006 &  &  & 0.467$^{**}$ \\ 
  &  &  & (0.236) \\ 
  & & & \\ 
 Percentage Black x 2006 &  &  & 0.064 \\ 
  &  &  & (0.093) \\ 
  & & & \\ 
 Unemployment Rate x 2006 &  &  & $-$0.001 \\ 
  &  &  & (0.007) \\ 
  & & & \\ 
 Annual Average Pay x 2006 &  &  & 0.003 \\ 
  &  &  & (0.002) \\ 
  & & & \\ 
 Constant & $-$0.087 & $-$0.057 & 0.019 \\ 
  & (0.098) & (0.099) & (0.098) \\ 
  & & & \\ 
\hline \\[-1.8ex] 
Observations & 34,359 & 34,359 & 34,359 \\ 
R$^{2}$ & 0.385 & 0.422 & 0.418 \\ 
Adjusted R$^{2}$ & 0.385 & 0.422 & 0.417 \\ 
\hline 
\hline \\[-1.8ex] 
\textit{Note:}  & \multicolumn{3}{r}{$^{*}$p$<$0.1; $^{**}$p$<$0.05; $^{***}$p$<$0.01} \\ 
\end{tabular} 
\end{table}


\end{document}
