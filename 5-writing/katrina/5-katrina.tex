\documentclass[]{article}
\usepackage{lmodern}
\usepackage{setspace}
\setstretch{2}
\usepackage{amssymb,amsmath}
\usepackage{ifxetex,ifluatex}
\usepackage{fixltx2e} % provides \textsubscript
\ifnum 0\ifxetex 1\fi\ifluatex 1\fi=0 % if pdftex
  \usepackage[T1]{fontenc}
  \usepackage[utf8]{inputenc}
\else % if luatex or xelatex
  \ifxetex
    \usepackage{mathspec}
  \else
    \usepackage{fontspec}
  \fi
  \defaultfontfeatures{Ligatures=TeX,Scale=MatchLowercase}
\fi
% use upquote if available, for straight quotes in verbatim environments
\IfFileExists{upquote.sty}{\usepackage{upquote}}{}
% use microtype if available
\IfFileExists{microtype.sty}{%
\usepackage{microtype}
\UseMicrotypeSet[protrusion]{basicmath} % disable protrusion for tt fonts
}{}
\usepackage[margin=1in]{geometry}
\usepackage{hyperref}
\hypersetup{unicode=true,
            pdftitle={The Effect of Disasters on Migration Destinations: Evidence from Hurricane Katrina},
            pdfauthor={Jonathan Eyer, Adam Rose, Noah Miller; and Robert Dinterman},
            pdfborder={0 0 0},
            breaklinks=true}
\urlstyle{same}  % don't use monospace font for urls
\usepackage{longtable,booktabs}
\usepackage{graphicx,grffile}
\makeatletter
\def\maxwidth{\ifdim\Gin@nat@width>\linewidth\linewidth\else\Gin@nat@width\fi}
\def\maxheight{\ifdim\Gin@nat@height>\textheight\textheight\else\Gin@nat@height\fi}
\makeatother
% Scale images if necessary, so that they will not overflow the page
% margins by default, and it is still possible to overwrite the defaults
% using explicit options in \includegraphics[width, height, ...]{}
\setkeys{Gin}{width=\maxwidth,height=\maxheight,keepaspectratio}
\IfFileExists{parskip.sty}{%
\usepackage{parskip}
}{% else
\setlength{\parindent}{0pt}
\setlength{\parskip}{6pt plus 2pt minus 1pt}
}
\setlength{\emergencystretch}{3em}  % prevent overfull lines
\providecommand{\tightlist}{%
  \setlength{\itemsep}{0pt}\setlength{\parskip}{0pt}}
\setcounter{secnumdepth}{0}
% Redefines (sub)paragraphs to behave more like sections
\ifx\paragraph\undefined\else
\let\oldparagraph\paragraph
\renewcommand{\paragraph}[1]{\oldparagraph{#1}\mbox{}}
\fi
\ifx\subparagraph\undefined\else
\let\oldsubparagraph\subparagraph
\renewcommand{\subparagraph}[1]{\oldsubparagraph{#1}\mbox{}}
\fi

%%% Use protect on footnotes to avoid problems with footnotes in titles
\let\rmarkdownfootnote\footnote%
\def\footnote{\protect\rmarkdownfootnote}

%%% Change title format to be more compact
\usepackage{titling}

% Create subtitle command for use in maketitle
\newcommand{\subtitle}[1]{
  \posttitle{
    \begin{center}\large#1\end{center}
    }
}

\setlength{\droptitle}{-2em}
  \title{The Effect of Disasters on Migration Destinations: Evidence from
Hurricane Katrina}
  \pretitle{\vspace{\droptitle}\centering\huge}
  \posttitle{\par}
  \author{Jonathan Eyer, Adam Rose, Noah Miller\footnote{University of Southern
  California} \\ and Robert Dinterman\footnote{The Ohio State University}}
  \preauthor{\centering\large\emph}
  \postauthor{\par}
  \predate{\centering\large\emph}
  \postdate{\par}
  \date{16 June 2017}

\usepackage{amsmath}

\begin{document}
\maketitle
\begin{abstract}
While post-disaster migration can move vulnerable populations from
dangerous regions to relatively safe ones, little is known about the
processes by which migrants select new homes. We utilize an econometric
model of migrant flows to examine the characteristics of the
destinations that attracted migrants leaving the New Orleans area
following Hurricane Katrina in 2005 relative to migration behaviors in
other years. We find an increase flow of migrants to large, nearby
counties with a mixed effect of economic variables on migration. We
generally do not find changes in the overall mix of migration, however,
suggesting that disasters increase the magnitude of migration but not
its direction.
\end{abstract}

\newpage

\section{Introduction}\label{introduction}

Natural disasters can cause widespread destruction and weaken local
economies. These impacts can lead to permanent migration away from
disaster-affected areas. Such permanent, or even semi-permanent,
migration induced by natural disasters has the potential to
significantly reshape the distribution of national and global
populations and economies (McIntosh 2008). Moreover, because migration
moves people out of the path of some disasters, and potentially into the
path of other ones, post-disaster migration has implications for the
risks associated with future events (Gráda and O'Rourke 1997). Finally,
the migration itself and the loss of community cohesion suggests the
need for consideration of mental health support in communities that will
receive large numbers of disaster migrants (Weber and Peek 2012).

Migration following disasters is well-documented for major events like
Hurricane Katrina, the 2011 Tohoku earthquake/tsunami and subsequent
Fukushima nuclear disaster, and the 2004 Indian Ocean tsunami. Because
of the pressures placed on the affected population, disasters can cause
migration among a wider portion of the population than those who migrate
normally -- i.e.~in a non-disaster related context (Gray et al. 2014).
While the propensity for disaster-affected populations to migrate is
documented, less is known about the preferences that impact the
destination of disaster-affiliated migrants.

Influences on migration decisions are generally framed in the context of
``push'' factors and ``pull'' factors. Push factors cause people to want
to leave the origin while pull factors cause people to want to go to a
specific destination. For example, high unemployment in the origin
signals poor job prospects and is seen as a push factor for
out-migration. Similarly, a low cost of living might pull people toward
a particular destination.

The circumstances of a disaster, however, may shift the relative
importance of various pull factors in attracting migrants. This could
occur because preferences over these pull factors are dependent on the
state in which the decision to migrate is made. \emph{Alternatively, a
disaster induces traditional non-migrants to move, and, if these people
have different preferences than those who are traditional movers, then
the overall destination of post-disaster migrants will shift.}

An understanding of these pull factors is important for crafting natural
disaster recovery policies, understanding the likely evolution of
disaster damages, and evaluating the prospects for repatriation. For
example, if post-disaster migrants are credit-constrained and unable to
move to the optimal location, government subsidies for relocation costs
might be justified. If post-disaster migrants move to other areas that
are at high risk of natural disasters, then the relocation costs will
not mitigate future disaster losses and may actually increase future
social costs related to natural disasters.

Hurricane Katrina, which struck New Orleans in 2005, provides an ideal
case study to examine the factors that influence the destination of
disaster migrants. Most residents of New Orleans evacuated prior to the
Hurricane, and following the storm most remaining residents were
evacuated by the Federal Emergency Management Agency (FEMA). All
together, approximately 1.5 million people evacuated the New Orleans
area which accounted for approximately 96\% of New Orleans residents and
80\% of residents surrounding the city (Groen and Polivka 2008; Elliott
and Pais 2006). While a large number of evacuees were initially
relocated to Houston by FEMA, Katrina evacuees relocated throughout the
country. Nearly every state received FEMA funding for costs associated
with supporting evacuees from Katrina.

Permanent migrants -- as opposed to migrants that evenentually returned
to the New Orleans area -- were generally younger, more likely to have
children, and more likely to be black (Groen and Polivka 2010). There
was also an increased flow of migrants from neighboring communities in
the years following Katrina compared to the years prior to Katrina
indicating that the mirants which relocated to nearby communities were
more likely to return than those further away (Fussell, Curtis, and
DeWaard 2014).

In this paper, we examine the migration pull factors in terms of
characteristics of the destinations of post-Katrina migration out of the
New Orleans area by using data on the movement of IRS return filings
between counties and a range of county-destination attributes. This
paper contributes to the literature by estimating the relative
importance of a range of factors in post-disaster relocation decisions.
This work conveys a range of policy implications surrounding disasters
and climate change. By identifying the characteristics that draw
migrants following natural disasters, we increase the understanding of
future migration patterns as disasters grow more frequent. Our
consideration of distance in the relocation decision also highlights the
extent to which post-disaster migrants will be removed from similarly
disaster-prone areas.

The rest of the paper proceeds as follows. In Section \ref{sec:theory},
we review the theoretical structure of migration decisions, in Section
\ref{sec:data} discuss our data sources, in Section \ref{sec:methods}
present our estimating equations, and in Section \ref{sec:results}
present our results.

\section{\texorpdfstring{Conceptual Underpinnings
\label{sec:theory}}{Conceptual Underpinnings }}\label{conceptual-underpinnings}

From an economic standpoint, migration decisions are based on households
comparing their expected lifetime utility in their current location (the
origin) to a location to which they could move (the destination)
(Greenwood 1985; Greenwood 1975). Yun and Waldorf (2016) examine the
decision about whether or not to migrate in an expected lifetime utility
framework and focus on the extent to which Katrina induced migration by
those who would not otherwise have migrated. The utility that a
household expects to receive from living in a particular location
depends on economic variables such as the wages and cost-of-living
associated with an area, but also on non-economic variables such as
environmental amenities, family and social ties, and perceptions about
safety. A household will decide to migrate if the increase in expected
lifetime utility obtained by moving from the origin to the destination
exceeds the costs of moving. These costs include the financial costs
associated with moving, as well as more abstract factors such as the
social costs incurred by the move.

The decision to migrate is generally endogenous to migrant
characteristics. Highly-skilled migrants who expect to receive large
wage premiums are more likely to migrate than low-skilled workers
(Borjas 1987). Similarly, migration is costly, Chiswick (1999) notes
that those who are less credit-constrained are more able to afford the
upfront costs associated with an optimal relocation decision.

Natural disasters, however, cause exogenous variation in the expected
lifetime utility at the origin. For example, property damage would
require repair costs in order to stay at the origin, and a weakened
local economy would lower wages at the origin. Similarly, if a disaster
causes households to update their beliefs about the likelihood and
severity of subsequent events, this could lower the expected utility of
remaining in the origin. These effects would cause households to
re-evaluate their location decisions and potentially choose to migrate
due to the decreased expected life-time utility at their origin (Yun and
Waldorf 2016).

In the event of major natural disasters like Hurricane Katrina and the
Fukushima nuclear disaster, the push factors are relatively obvious;
people leave the origin because of mandatory evacuation requirements,
legal inability to return due to quarantines, loss of employment
opportunities, etc. It is less obvious what draws migrants to particular
locations following a disaster. One might be particularly concerned that
post-disaster migrants are systematically different than those who
choose to migrate under other circumstances. Disaster-related migrants,
for example, might feel compelled to relocate more quickly or have less
wealth with which to bear moving costs. Hence, they may not move to
optimal locations in comparison to normal circumstances, or what Yun and
Waldorf (2016) refer to as ``double-victimization.'' Black et al. (2011)
suggest that population movements due to disasters are typically short
distance, though this conclusion seems to be counter to what happened in
the aftermath of Hurricane Katrina.

Several variables have been suggested, and some tested, to explain the
pull factors. Many of these are traditional in the gravity model
literature of migration, such as wage and cost-of-living differentials,
distance, moving costs, and general economic health of the destination
(Borjas 1987; Rupasingha, Liu, and Partridge 2015). Broadening the
analysis leads to consideration of amenities, family ties, racial/ethnic
affinities, migration networks, and institutions (McKenzie and Rapoport
2010; Nifo and Vecchione 2014). The destination choice itself is
dependent on the reason that drives the individual to migrate (Findlay
2011). One might conclude that short or long-run hazard vulnerability
would be major considerations, but Black et al. (2011) and Fielding
(2011) emphasize the primacy of socioeconomic over environmental
variables in current migration decisions, though on the basis of only
anecdotal information.

\emph{Transition needed here}

\section{\texorpdfstring{Data \label{sec:data}}{Data }}\label{data}

Our primary source of data is the Internal Revenue Service (IRS)
Statistics of Income Division's migration data (Internal Revenue Service
2017). These data are based on year-to-year address changes reported on
individual income tax returns filed with the IRS and aggregated up to
the county level beginning in 1990. The data reports county-to-county
flows of households, people, and income as proxied by number of returns
filed, number of personal exemptions claimed, and total adjusted gross
income. The county-to-county flows can be seen as either inflows or
outflows depending on the county of interest. The IRS suppresses
observations with fewer than 10 filers due to disclosure concerns and
prior to 2004 the IRS did not distinguish between a non-disclosed
observation and a true 0 observation. Because we cannot distinguish
between a county-to-county pair which received between 1 and 9 filers
from a county-to-county pair which did not receive any filers, we treat
these potential non-disclosed counties as 0.

Given our focus on New Orleans, we restrict our interest to outflow
migration from the parishes most severely affected by Hurricane Katrina
in 2005. We define the population affected by Hurricane Katrina as those
residing in Jefferson, Lafourche, Orleans, Plaquemines, St.~Bernard,
St.~Charles, St.~John The Baptist, St.~Tammany, and Terrebonne
Parishes.\footnote{While Louisiana is organized into parishes rather
  than counties, we will use the term counties throughout this paper to
  facilitate discussion of destination locations.} These parishes
constitute the New Orleans metro, as well as two surrounding Parishes
which adjoin the metro area. While there was some migration between
affected regions (i.e., moving from a county that was severely affected
to one that was slightly less affected), we remove these migrants from
our sample to facilitate a simpler interpretation of outflow migrants.

We aggregate annual migration flows to each destination county across
the 9 highly-affected origin counties between 2000 and 2010. The result
is an 11-year panel of population flows to the 3,095 destination
counties.\footnote{There are 3,144 counties and county equivalents in
  the U.S. and affected counties are removed from the set of potential
  destination counties as well as any counties for which explanatory
  variables are unavailable.} There is a non-zero number of migrants to
approximately 5.4\% of the county-year observations in our dataset. In
2005, however, 13.8\% of US counties received migrants from the affected
area. With the exception of 2005, the number of migrants and the
proportion of counties that receive migrants from New Orleans is
relatively consistent over time.

Most migrants from the New Orleans area move to counties that are
relatively close. Table \ref{tab:commondeststate} and
\ref{tab:commondest} present the states and counties that received the
greatest proportion of migrants from the New Orleans area in 2005. Each
of the five states that received the most migrants from New Orleans area
in the South, as are the ten most common destination counties. With the
exception of Tangiphoa and Ascension parishes in Louisiana, each of the
most common destination counties are within a metropolitan area.

We supplement the IRS migration data with a number of explanatory
variables that might affect the relative attractiveness of a destination
county. Unemployment rates are obtained from the Bureau of Labor
Statistics (BLS) which reports annual labor force data by county (Bureau
of Labor Statistics 2017a). This data includes the number of people in
the labor force as well as the number of unemployed people, and
unemployment rates are calculated from these values. Average annual wage
data are obtained from the Quarterly Census of Wages provided by the BLS
(Bureau of Labor Statistics 2017b). Both of these variables proxy for
the labor market of a given county with the availability of jobs and
their relative pay. Median monthly rents for 2-bedroom units are
obtained from the Department of Housing and Urban Development
(Department of Housing and Urban Development 2017). Rents are a measure
of cost-of-living for a county. And finally, the metropolitan
classification of each county is denoted using the rural-urban continuum
codes from the United States Department of Agriculture Economic Research
Services (United States Department of Agriculture Economic Research
Service 2017).

Summary statistics for the relevant variables are provided in Table
\ref{tab:sumstats}.

\section{\texorpdfstring{Methods
\label{sec:methods}}{Methods }}\label{methods}

In order to understand how Hurricane Katrina affected migration we
estimate a series of models of migration outflow from the affected
counties. We adopt a traditional gravity model as our baseline model:

\[ Y_{i,t} = \alpha + \gamma {D}_{i} + \beta_1 {P}_{i,t} + \beta_2 {Katrina}_{t} + \mathbf{\beta_3} \mathbf{X_{i,t}} + \varepsilon_{i,t} \label{eq:basereg} \]

where \(i\) indicates destination county, \(t\) denotes the year of
interest, \(Y_{i,t}\) is our dependent variable which captures migration
flows from our previously defined New Orleans area, \({D}_{i}\) is the
Euclidean distance from the centroid of a county to the affected area,
\({P}_{i,t}\) is a measure of population, \({Katrina}_{t}\) is an
indicator for if Katrina landed in that year, and \(\mathbf{X_{i,t}}\)
contains the relevant economic explanatory variables for destination
county: unemployment rate, average annual wages, average monthly rent,
and an indicator of metropolitan status of the county.

We consider a set of dependent variables for our regressions, and
estimate the model separately with each potential dependent variable.
First, we consider the number of migrants to county \(i\) itself. Next,
we consider the inverse hyperbolic sine of migrants to county \(i\),
which is comparable to the natural log and yields semi-elasticities.
Next, because many counties receive no migrants at all we estimate a
linear probability model in which \(Y_{i,t}\) takes on a value of one if
any migrants are observed moving to a particular county in a given year.
Finally, we calculate the share of New Orleans' area migration that goes
to each county.

Of particular interest is the coefficient associated with the
\({Katrina}_{t}\) variable which would indicate how the out-migration
was affected by the natural disaster. The first two variables capture
the intensive margin of migration, while the third measures the
extensive margin. These models describe the distribution and magnitude
of post-disaster migration. The final specification speaks to the mix of
migration across potential destinations, holding constant the magnitude
of migration flows.

In addition to the baseline regressions, we like to better understand
the effect of Hurricane Katrina on disaster-related migration. In order
to do so we interact each of our explanatory variables with our
\({Katrina}_{t}\) variables:

\begin{equation} \label{eq:intreg}
\begin{split}
Y_{i,t} = &\alpha + \gamma {D}_{i} + \beta_1 {P}_{i,t} + \beta_2 {Katrina}_{t} + \mathbf{\beta_3} \mathbf{X_{i,t}} + \gamma_k {D}_{i} \times {Katrina}_{t} + \\
& \beta_{1k} {P}_{i,t} \times {Katrina}_{t} + \mathbf{\beta_{3k}} \mathbf{X_{i,t}} \times {Katrina}_{t} + \varepsilon_{i,t} 
\end{split}
\end{equation}

The interaction terms capture the change in preferences over each pull
factor variable relative to the other years in our sample, when a major
disaster did not strike New Orleans. If Hurricane Katrina shifted the
relative importance of pull factors in the destination-selection
process, we would expect these interaction terms to be statistically
significant. Similarly, if these interaction terms are statistically
indistinguishable from zero it suggests that migration was no different
in 2005, for example, than it was in years that were not affected by
Hurricane Katrina.

\section{\texorpdfstring{Results
\label{sec:results}}{Results }}\label{results}

In Table \ref{reg:regmain}, we present the results of a series of OLS
regressions related to the flow of migrants from the New Orleans area.
In each column, we present a particular transformation of the flow of
migrants from New Orleans to each destination county. Columns 1 and 2
correspond to the count of migrants and the inverse hyperbolic sine of
migrant count, Column 3 relates to a dummy variable for whether or not a
county had more than 10 migrants from any of the affected counties, and
Column 4 is each county's share of all migrants leaving the New Orleans
area.

In general, the results reflect relatively standard migration
preferences. Counties that are large, or close to the New Orleans area
are more likely to receive migrants than less populous counties or those
that are far from southern Louisiana. Similarly, counties with lower
unemployment and higher wages are more likely to receive migrants than
counties with less robust economies, although we find no statistically
significant effect of a destination county's median rent on migration
decisions. These effects are each true across each of the
specifications.

When focusing on the intensive margin in 2005 (Columns 1 and 2), we
primarily find an increased penalty on counties that are distant from
New Orleans. While each additional hundred kilometers of distance
resulted in a 2.2\% decrease in the number of migrants to a county in
the baseline, in 2005 the same marginal change in distance resulted in a
7.5\% drop in migrants. In the inverse hyperbolic sine specification, we
find impacts on the economic pull factors that are statistically
distinct from their non-interacted counterparts. The interpretation of
the changes in these economic variables are mixed. The positive
coefficients on the unemployment rate and on median rent each indicate a
reduction in the importance of economic considerations, although the
positive impact on average pay suggests the opposite. There is also more
migration towards larger counties relative to other years; a county with
an additional million residents would receive 217\% more migrants than a
county with fewer residents in most years but in 2005 the differential
would be closer to 300\%. We find similar effects on the extensive
margin (Column 3) in 2005. Distance and population are more important in
determining whether or not a county receives any migrants from the New
Orleans area, while there is mixed evidence of a differential impact
from economic considerations.

Across both the intensive and extensive margins, migration preferences
in 2006 generally return to their baseline level. There is still a
differential impact of distance, indicating that more migrants from the
New Orleans area in 2006 moved to counties that were close to New
Orleans than in other years, although the effect is much smaller than it
was in 2005. There is also some evidence of increased migration towards
counties with high-paying jobs relative to other years. These two
effects are generally consistent with the idea that those who stayed in
New Orleans in 2005 placed greater concern on economic considerations
than those who migrated away in 2005, and less concern on relocating
cheaply. One possible explanation for this effect is that those who left
New Orleans in 2005 did so because of the city's weakened post-Katrina
economy.

The results are less clear when we focus on explaining the share of
migrants who moved to each county (Column 4). In most years, the share
of migrants tends to reflect the results from the intensive and
extensive margin regressions. Large counties that are close to New
Orleans that have strong economies receive a greater share of migrants
than distant, small counties with weak economies. When we focus on the
interaction terms, however, the only discernible change is in the effect
of the unemployment rate. As with the other three regressions, and
consistent with the idea that economic conditions matter less in the
face of a natural disaster, the unemployment rate mattered less in 2005
than it did in other years. While a one percentage point increase in the
unemployment rate reduced the share of New Orleans migrants to a county
by about 0.2 percentage points, in 2005 counties with higher
unemployment rates actually received a greater share of migrants from
New Orleans than those with low unemployment rates. There is no
statistically significant shift of migrants from distant counties
towards close ones or from large counties to smaller ones. Furthermore,
none of the interacted explanatory variables in 2006 are statistically
different from zero, indicating that the overall mix of migrants was
unchanged in the aftermath of Hurricane Katrina.

In summary, while Hurricane Katrina resulted in substantial increases in
outflow migration, much of it to areas that were relatively likely to be
affected by future hurricanes, the overall composition of migration
remained relatively unchanged. Because people tend to migrate to close
areas rather than distant ones, post-disaster migration is unlikely to
be a panacea for natural hazards. Still, disaster-related migration may
still result in some social benefits. People tend to migrate towards
urban population centers rather than rural communities, so disasters may
serve to accelerate the shift of populations towards cities, in which
they may benefit from agglomeration effects and experience higher
productivity.

\section{\texorpdfstring{Conclusion
\label{sec:conclusion}}{Conclusion }}\label{conclusion}

This study has focused on the characteristics of the destinations of
post-Katrina migrants from the New Orleans area using IRS data on the
movement of tax exemptions in order to explain the pattern of
out-migration. An understanding of the factors driving post-disaster
migration is important both in planning for shifts in population and in
assessing future damages from natural disasters.

In most years, migration away from the New Orleans area corresponds with
traditional gravity model results. Migrants prefer close destinations to
distant ones, and tend towards large, economically strong counties
rather than rural ones with fewer economic prospects. This is true
across a range of specifications describing outflow migration.

As natural disasters grow more frequent and more costly,
disaster-related migration will increase. While migration away from
high-risk regions could reduce future disaster losses, migrants tend to
move to counties that suffer similar risk to the ones that they are
leaving. Government policy could be used to incentive migration towards
safer destinations that are further from the affected area, but in the
absence of much policy interventions, migration is unlikely to lower the
costs of future disasters.

\newpage

\section{References}\label{references}

\hypertarget{refs}{}
\hypertarget{ref-black2011effect}{}
Black, Richard, W Neil Adger, Nigel W Arnell, Stefan Dercon, Andrew
Geddes, and David Thomas. 2011. ``The Effect of Environmental Change on
Human Migration.'' \emph{Global Environmental Change} 21. Elsevier:
S3--S11.

\hypertarget{ref-borjas1987self}{}
Borjas, George J. 1987. ``Self-Selection and the Earnings of
Immigrants.'' \emph{American Economic Review} 77 (4): 531--53.

\hypertarget{ref-blsdata}{}
Bureau of Labor Statistics. 2017a. ``Local Area Unemployment
Statistics.''

\hypertarget{ref-qcewdata}{}
---------. 2017b. ``Quarterly Census of Employment and Wages.''

\hypertarget{ref-chiswick1999immigrants}{}
Chiswick, Barry R. 1999. ``Are Immigrants Favorably Self-Selected?''
\emph{American Economic Review} 89 (2). JSTOR: 181--85.

\hypertarget{ref-hudrentdata}{}
Department of Housing and Urban Development. 2017. ``Fair Market
Rents.''

\hypertarget{ref-elliott2006race}{}
Elliott, James R, and Jeremy Pais. 2006. ``Race, Class, and Hurricane
Katrina: Social Differences in Human Responses to Disaster.''
\emph{Social Science Research} 35 (2). Elsevier: 295--321.

\hypertarget{ref-fielding2011impacts}{}
Fielding, AJ. 2011. ``The Impacts of Environmental Change on Uk Internal
Migration.'' \emph{Global Environmental Change} 21. Elsevier:
S121--S130.

\hypertarget{ref-findlay2011migrant}{}
Findlay, Allan M. 2011. ``Migrant Destinations in an Era of
Environmental Change.'' \emph{Global Environmental Change} 21. Elsevier:
S50--S58.

\hypertarget{ref-fussell2014recovery}{}
Fussell, Elizabeth, Katherine J Curtis, and Jack DeWaard. 2014.
``Recovery Migration to the City of New Orleans After Hurricane Katrina:
A Migration Systems Approach.'' \emph{Population and Environment} 35
(3). Springer: 305--22.

\hypertarget{ref-gray2014studying}{}
Gray, Clark, Elizabeth Frankenberg, Thomas Gillespie, Cecep Sumantri,
and Duncan Thomas. 2014. ``Studying Displacement After a Disaster Using
Large-Scale Survey Methods: Sumatra After the 2004 Tsunami.''
\emph{Annals of the Association of American Geographers} 104 (3). Taylor
\& Francis: 594--612.

\hypertarget{ref-grada1997migration}{}
Gráda, Cormac Ó, and Kevin H O'Rourke. 1997. ``Migration as Disaster
Relief: Lessons from the Great Irish Famine.'' \emph{European Review of
Economic History} 1 (1). Oxford University Press: 3--25.

\hypertarget{ref-greenwood1975research}{}
Greenwood, Michael J. 1975. ``Research on Internal Migration in the
United States: A Survey.'' \emph{Journal of Economic Literature}. JSTOR,
397--433.

\hypertarget{ref-greenwood1985human}{}
---------. 1985. ``Human Migration: Theory, Models, and Empirical
Studies.'' \emph{Journal of Regional Science} 25 (4). Wiley Online
Library: 521--44.

\hypertarget{ref-groen2008hurricane}{}
Groen, Jeffrey A, and Anne E Polivka. 2008. ``Hurricane Katrina
Evacuees: Who They Are, Where They Are, and How They Are Faring.''
\emph{Monthly Labor Review} 131. HeinOnline: 32.

\hypertarget{ref-groen2010going}{}
---------. 2010. ``Going Home After Hurricane Katrina: Determinants of
Return Migration and Changes in Affected Areas.'' \emph{Demography} 47
(4). Springer: 821--44.

\hypertarget{ref-irsmigdata}{}
Internal Revenue Service. 2017. ``Statistics of Income Tax Stats -
Migration Data.''

\hypertarget{ref-mcintosh2008measuring}{}
McIntosh, Molly Fifer. 2008. ``Measuring the Labor Market Impacts of
Hurricane Katrina Migration: Evidence from Houston, Texas.''
\emph{American Economic Review} 98 (2). JSTOR: 54--57.

\hypertarget{ref-mckenzie2010self}{}
McKenzie, David, and Hillel Rapoport. 2010. ``Self-Selection Patterns in
Mexico-Us Migration: The Role of Migration Networks.'' \emph{Review of
Economics and Statistics} 92 (4). MIT Press: 811--21.

\hypertarget{ref-nifo2014institutions}{}
Nifo, Annamaria, and Gaetano Vecchione. 2014. ``Do Institutions Play a
Role in Skilled Migration? The Case of Italy.'' \emph{Regional Studies}
48 (10). Taylor \& Francis: 1628--49.

\hypertarget{ref-rupasingha2015rural}{}
Rupasingha, Anil, Yongzheng Liu, and Mark Partridge. 2015. ``Rural
Bound: Determinants of Metro to Non-Metro Migration in the United
States.'' \emph{American Journal of Agricultural Economics} 97 (3).
Oxford University Press: 680--700.

\hypertarget{ref-ersruuc}{}
United States Department of Agriculture Economic Research Service. 2017.
``Rural-Urban Continuum Codes.''

\hypertarget{ref-weber2012displaced}{}
Weber, Lynn, and Lori A Peek. 2012. \emph{Displaced: Life in the Katrina
Diaspora}. University of Texas Press.

\hypertarget{ref-yun2016day}{}
Yun, Seong Do, and Brigitte S Waldorf. 2016. ``The Day After the
Disaster: Forced Migration and Income Loss After Hurricanes Katrina and
Rita.'' \emph{Journal of Regional Science} 56 (3). Wiley Online Library:
420--41.

\clearpage

\section{Tables and Figures}\label{tables-and-figures}

\begin{longtable}[]{@{}lrl@{}}
\caption{Most Common State Destination for Migrants in 2005
\label{tab:commondeststate}}\tabularnewline
\toprule
State & Migrants & Percentage of Total\tabularnewline
\midrule
\endfirsthead
\toprule
State & Migrants & Percentage of Total\tabularnewline
\midrule
\endhead
Texas & 73,252 & 40.3\%\tabularnewline
Louisiana & 45,014 & 24.8\%\tabularnewline
Georgia & 14,480 & 7.96\%\tabularnewline
Mississippi & 10,327 & 5.68\%\tabularnewline
Florida & 5,951 & 3.27\%\tabularnewline
\bottomrule
\end{longtable}

\clearpage

\begin{longtable}[]{@{}lllrl@{}}
\caption{Most Common County Destination for Migrants in 2005
\label{tab:commondest}}\tabularnewline
\toprule
FIPS & County & State & Migrants & Percentage of Total\tabularnewline
\midrule
\endfirsthead
\toprule
FIPS & County & State & Migrants & Percentage of Total\tabularnewline
\midrule
\endhead
48201 & Harris & Texas & 38,033 & 20.9\%\tabularnewline
22033 & East Baton Rouge & Louisiana & 14,291 & 7.86\%\tabularnewline
48113 & Dallas & Texas & 9,971 & 5.48\%\tabularnewline
48439 & Tarrant & Texas & 5,575 & 3.07\%\tabularnewline
22105 & Tangipahoa & Louisiana & 4,758 & 2.62\%\tabularnewline
22055 & Lafayette & Louisiana & 3,377 & 1.86\%\tabularnewline
48029 & Bexar & Texas & 3,114 & 1.71\%\tabularnewline
22005 & Ascension & Louisiana & 2,912 & 1.6\%\tabularnewline
13089 & Dekalb & Georgia & 2,783 & 1.53\%\tabularnewline
47157 & Shelby & Tennessee & 2,769 & 1.52\%\tabularnewline
\bottomrule
\end{longtable}

\clearpage

\begin{longtable}[]{@{}lrrrr@{}}
\caption{Summary Statistics \label{tab:sumstats}}\tabularnewline
\toprule
variable & mean & sd & min & max\tabularnewline
\midrule
\endfirsthead
\toprule
variable & mean & sd & min & max\tabularnewline
\midrule
\endhead
distance & 13.6623594 & 7.7881486 & 0.5732867 & 69.167293\tabularnewline
exmpt\_katrina & 11.6515494 & 248.5658002 & 0.0000000 &
38033.000000\tabularnewline
fmr & 5.5964191 & 1.7244501 & 3.2300000 & 19.400000\tabularnewline
metro & 0.3476575 & 0.4762336 & 0.0000000 & 1.000000\tabularnewline
pay & 29.5532491 & 7.2867086 & 0.0000000 & 101.084000\tabularnewline
population & 0.0740929 & 0.2331340 & 0.0000000 & 7.421764\tabularnewline
un\_rate & 6.0421604 & 2.6590955 & 1.3672343 & 28.840018\tabularnewline
\bottomrule
\end{longtable}

\clearpage

\clearpage
\scriptsize

\begin{table}[!htbp] \centering 
  \caption{\label{reg:regmain}Effect of Destination Characteristics on New Orleans Outflow Migration} 
  \label{} 
\scriptsize 
\begin{tabular}{@{\extracolsep{5pt}}lcccc} 
\\[-1.8ex]\hline 
\hline \\[-1.8ex] 
 & \multicolumn{4}{c}{\textit{Dependent variable:}} \\ 
\cline{2-5} 
 & Flow & IHS & LP & Share \\ 
\\[-1.8ex] & (1) & (2) & (3) & (4)\\ 
\hline \\[-1.8ex] 
 Population (Millions) & 95.327$^{**}$ & 2.167$^{***}$ & 0.400$^{***}$ & 0.455$^{**}$ \\ 
  & (39.123) & (0.496) & (0.102) & (0.187) \\ 
  & & & & \\ 
 Distance (Hundreds of Kilometers) & $-$1.036$^{***}$ & $-$0.022$^{***}$ & $-$0.004$^{***}$ & $-$0.005$^{***}$ \\ 
  & (0.242) & (0.003) & (0.001) & (0.001) \\ 
  & & & & \\ 
 Unemployment Rate & $-$0.492$^{***}$ & $-$0.018$^{***}$ & $-$0.004$^{***}$ & $-$0.002$^{***}$ \\ 
  & (0.150) & (0.003) & (0.001) & (0.001) \\ 
  & & & & \\ 
 Average Pay & 0.391$^{*}$ & 0.015$^{***}$ & 0.003$^{***}$ & 0.002$^{**}$ \\ 
  & (0.203) & (0.004) & (0.001) & (0.001) \\ 
  & & & & \\ 
 Median Rent & $-$1.285 & 0.010 & 0.003 & $-$0.006 \\ 
  & (1.362) & (0.015) & (0.003) & (0.006) \\ 
  & & & & \\ 
 Non Metro & 2.779 & $-$0.057 & $-$0.017$^{*}$ & 0.014 \\ 
  & (3.143) & (0.048) & (0.010) & (0.015) \\ 
  & & & & \\ 
 Year 2005 & 79.976 & $-$1.107$^{***}$ & $-$0.217$^{***}$ & 0.005 \\ 
  & (174.558) & (0.237) & (0.042) & (0.078) \\ 
  & & & & \\ 
 Population X 2005 & 1,062.432 & 0.668$^{**}$ & 0.001 & 0.180 \\ 
  & (783.099) & (0.338) & (0.035) & (0.265) \\ 
  & & & & \\ 
 Distance X 2005 & $-$7.544$^{***}$ & $-$0.053$^{***}$ & $-$0.009$^{***}$ & 0.0003 \\ 
  & (1.737) & (0.004) & (0.001) & (0.001) \\ 
  & & & & \\ 
 Unemployment Rate X 2005  & 10.140$^{**}$ & 0.111$^{***}$ & 0.020$^{***}$ & 0.007$^{***}$ \\ 
  & (3.984) & (0.017) & (0.003) & (0.002) \\ 
  & & & & \\ 
 Average Pay X 2005 & 3.064 & 0.027$^{***}$ & 0.004$^{***}$ & $-$0.00001 \\ 
  & (3.337) & (0.006) & (0.001) & (0.001) \\ 
  & & & & \\ 
 Median Rent X 2005 & $-$28.266 & 0.217$^{***}$ & 0.045$^{***}$ & $-$0.011 \\ 
  & (33.117) & (0.031) & (0.006) & (0.013) \\ 
  & & & & \\ 
 Non Metro X 2005 & 10.871 & $-$0.442$^{***}$ & $-$0.083$^{***}$ & $-$0.007 \\ 
  & (33.630) & (0.062) & (0.013) & (0.009) \\ 
  & & & & \\ 
 Population X 2006 & 60.817 & $-$0.214$^{**}$ & $-$0.060$^{**}$ & 0.035 \\ 
  & (56.251) & (0.106) & (0.029) & (0.113) \\ 
  & & & & \\ 
 Distance X 2006 & $-$0.702$^{***}$ & $-$0.006$^{***}$ & $-$0.001$^{***}$ & $-$0.0004 \\ 
  & (0.215) & (0.001) & (0.0003) & (0.0004) \\ 
  & & & & \\ 
 Unemployment Rate X 2006  & 0.462 & 0.002 & 0.0002 & 0.002 \\ 
  & (0.566) & (0.007) & (0.001) & (0.002) \\ 
  & & & & \\ 
 Average Pay X 2006 & 0.563$^{**}$ & 0.007$^{***}$ & 0.001$^{*}$ & 0.001 \\ 
  & (0.251) & (0.002) & (0.0005) & (0.001) \\ 
  & & & & \\ 
 Median Rent X 2006 & $-$2.228 & $-$0.023$^{**}$ & $-$0.003 & $-$0.006 \\ 
  & (2.214) & (0.011) & (0.002) & (0.005) \\ 
  & & & & \\ 
 Non Metro X 2006 & 1.780 & $-$0.022 & 0.0001 & $-$0.002 \\ 
  & (5.074) & (0.020) & (0.005) & (0.010) \\ 
  & & & & \\ 
 Constant & 10.583$^{*}$ & 0.008 & $-$0.012 & 0.044$^{*}$ \\ 
  & (5.754) & (0.109) & (0.022) & (0.025) \\ 
  & & & & \\ 
\hline \\[-1.8ex] 
Observations & 34,045 & 34,045 & 34,045 & 34,045 \\ 
R$^{2}$ & 0.113 & 0.338 & 0.312 & 0.115 \\ 
Adjusted R$^{2}$ & 0.113 & 0.338 & 0.312 & 0.115 \\ 
\hline 
\hline \\[-1.8ex] 
\textit{Note:}  & \multicolumn{4}{r}{$^{*}$p$<$0.1; $^{**}$p$<$0.05; $^{***}$p$<$0.01} \\ 
\end{tabular} 
\end{table}

\clearpage

\begin{longtable}[]{@{}lllll@{}}
\caption{\label{tab:reg_main}Effect of Destination County
Characteristics on New Orleans Outflow}\tabularnewline
\toprule
term & Flow & IHS & LP & Share\tabularnewline
\midrule
\endfirsthead
\toprule
term & Flow & IHS & LP & Share\tabularnewline
\midrule
\endhead
Intercept & 14.99 & -0.0308 & -0.0197 & 0.0458*\tabularnewline
& (12.7717) & (0.1154) & (0.023) & (0.0278)\tabularnewline
Population (Millions) & 193.0825* & 2.2278*** & 0.3989*** &
0.4739**\tabularnewline
& (110.3735) & (0.5207) & (0.1027) & (0.2188)\tabularnewline
Distance (Hundreds of Kilometers) & -1.8106*** & -0.0268*** & -0.0048***
& -0.005***\tabularnewline
& (0.4174) & (0.003) & (5e-04) & (0.0012)\tabularnewline
Unemployment Rate & -0.1306 & -0.0162*** & -0.0034*** &
-0.0015**\tabularnewline
& (0.2915) & (0.0035) & (7e-04) & (7e-04)\tabularnewline
Annual Pay (Thousands of USD) & 0.7592* & 0.0176*** & 0.0036*** &
0.002*\tabularnewline
& (0.4331) & (0.0047) & (9e-04) & (0.001)\tabularnewline
Median Monthly Rent & -3.9228 & 0.019 & 0.0056* & -0.0072\tabularnewline
& (3.7937) & (0.0155) & (0.0031) & (0.0076)\tabularnewline
Non-Metro & 4.1423 & -0.1107** & -0.0268*** & 0.0136\tabularnewline
& (7.3912) & (0.0499) & (0.0096) & (0.0168)\tabularnewline
Is 2005 & 51.6484*** & 0.5169*** & 0.092*** & -1e-04\tabularnewline
& (12.9228) & (0.025) & (0.005) & (0.0038)\tabularnewline
Adjusted R-Squared & 0.037 & 0.306 & 0.288 & 0.114\tabularnewline
Observations & 34,045 & 34,045 & 34,045 & 34,045\tabularnewline
\bottomrule
\end{longtable}

\clearpage

\begin{longtable}[]{@{}lllll@{}}
\caption{\label{tab:reg2005}Effect of Destination County Characteristics
on New Orleans Outflow - 2005 Interactions}\tabularnewline
\toprule
term & Flow & IHS & LP & Share\tabularnewline
\midrule
\endfirsthead
\toprule
term & Flow & IHS & LP & Share\tabularnewline
\midrule
\endhead
Intercept & 10.7085* & 0.0094 & -0.0115 & 0.0444*\tabularnewline
& (5.6626) & (0.1094) & (0.0225) & (0.025)\tabularnewline
Population (Millions) & 101.2034** & 2.1478*** & 0.3946*** &
0.4588**\tabularnewline
& (44.4983) & (0.4899) & (0.0996) & (0.1976)\tabularnewline
Distance (Hundreds of Kilometers) & -1.1058*** & -0.0226*** & -0.0041***
& -0.005***\tabularnewline
& (0.261) & (0.003) & (5e-04) & (0.0012)\tabularnewline
Unemployment Rate & -0.5351*** & -0.0178*** & -0.0037*** &
-0.0019***\tabularnewline
& (0.1671) & (0.0033) & (7e-04) & (7e-04)\tabularnewline
Annual Pay (Thousands of USD) & 0.4672** & 0.016*** & 0.0034*** &
0.002**\tabularnewline
& (0.2207) & (0.0045) & (9e-04) & (0.001)\tabularnewline
Median Monthly Rent & -1.5468 & 0.0073 & 0.0031 & -0.0062\tabularnewline
& (1.5563) & (0.0149) & (0.0031) & (0.0068)\tabularnewline
Non-Metro & 2.9913 & -0.0598 & -0.0172* & 0.014\tabularnewline
& (3.6052) & (0.0482) & (0.0096) & (0.0162)\tabularnewline
Is 2005 & 79.8503 & -1.1087*** & -0.2173*** & 0.0052\tabularnewline
& (174.7202) & (0.2368) & (0.0418) & (0.0784)\tabularnewline
Population x 2005 & 1056.5554 & 0.6868** & 0.0062 &
0.1761\tabularnewline
& (777.6244) & (0.3442) & (0.0371) & (0.2543)\tabularnewline
Distance x 2005 & -7.4741*** & -0.0526*** & -0.0087*** &
3e-04\tabularnewline
& (1.719) & (0.0039) & (8e-04) & (6e-04)\tabularnewline
Unemployment Rate x 2005 & 10.1832** & 0.1105*** & 0.0198*** &
0.0072***\tabularnewline
& (3.995) & (0.0167) & (0.003) & (0.0025)\tabularnewline
Pay x 2005 & 2.9878 & 0.0263*** & 0.0043*** & -1e-04\tabularnewline
& (3.3237) & (0.0061) & (0.0011) & (0.0013)\tabularnewline
Monthly Rent x 2005 & -28.0041 & 0.2196*** & 0.0449*** &
-0.01\tabularnewline
& (32.9195) & (0.0313) & (0.0062) & (0.0125)\tabularnewline
Non-Metro x 2005 & 10.6583 & -0.4387*** & -0.083*** &
-0.0066\tabularnewline
& (33.1141) & (0.062) & (0.0129) & (0.0081)\tabularnewline
Adjusted R-Squared & 0.112 & 0.337 & 0.311 & 0.115\tabularnewline
Observations & 34,045 & 34,045 & 34,045 & 34,045\tabularnewline
\bottomrule
\end{longtable}


\end{document}
