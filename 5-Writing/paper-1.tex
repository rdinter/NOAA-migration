\documentclass[]{article}
\usepackage{lmodern}
\usepackage{setspace}
\setstretch{2}
\usepackage{amssymb,amsmath}
\usepackage{ifxetex,ifluatex}
\usepackage{fixltx2e} % provides \textsubscript
\ifnum 0\ifxetex 1\fi\ifluatex 1\fi=0 % if pdftex
  \usepackage[T1]{fontenc}
  \usepackage[utf8]{inputenc}
\else % if luatex or xelatex
  \ifxetex
    \usepackage{mathspec}
  \else
    \usepackage{fontspec}
  \fi
  \defaultfontfeatures{Ligatures=TeX,Scale=MatchLowercase}
  \newcommand{\euro}{€}
\fi
% use upquote if available, for straight quotes in verbatim environments
\IfFileExists{upquote.sty}{\usepackage{upquote}}{}
% use microtype if available
\IfFileExists{microtype.sty}{%
\usepackage{microtype}
\UseMicrotypeSet[protrusion]{basicmath} % disable protrusion for tt fonts
}{}
\usepackage[margin=1in]{geometry}
\usepackage{hyperref}
\PassOptionsToPackage{usenames,dvipsnames}{color} % color is loaded by hyperref
\hypersetup{unicode=true,
            pdftitle={The Effect of Disasters on Migration},
            pdfauthor={Robert Dinterman; Jonathan Eyer, Noah Miller, and Adam Rose},
            pdfborder={0 0 0},
            breaklinks=true}
\urlstyle{same}  % don't use monospace font for urls
\usepackage{graphicx,grffile}
\makeatletter
\def\maxwidth{\ifdim\Gin@nat@width>\linewidth\linewidth\else\Gin@nat@width\fi}
\def\maxheight{\ifdim\Gin@nat@height>\textheight\textheight\else\Gin@nat@height\fi}
\makeatother
% Scale images if necessary, so that they will not overflow the page
% margins by default, and it is still possible to overwrite the defaults
% using explicit options in \includegraphics[width, height, ...]{}
\setkeys{Gin}{width=\maxwidth,height=\maxheight,keepaspectratio}
\setlength{\parindent}{0pt}
\setlength{\parskip}{6pt plus 2pt minus 1pt}
\setlength{\emergencystretch}{3em}  % prevent overfull lines
\providecommand{\tightlist}{%
  \setlength{\itemsep}{0pt}\setlength{\parskip}{0pt}}
\setcounter{secnumdepth}{0}

%%% Use protect on footnotes to avoid problems with footnotes in titles
\let\rmarkdownfootnote\footnote%
\def\footnote{\protect\rmarkdownfootnote}

%%% Change title format to be more compact
\usepackage{titling}

% Create subtitle command for use in maketitle
\newcommand{\subtitle}[1]{
  \posttitle{
    \begin{center}\large#1\end{center}
    }
}

\setlength{\droptitle}{-2em}
  \title{The Effect of Disasters on Migration}
  \pretitle{\vspace{\droptitle}\centering\huge}
  \posttitle{\par}
  \author{Robert Dinterman\footnote{North Caorlina State University} \\ Jonathan Eyer, Noah Miller, and Adam Rose\footnote{University of
  Southern California}}
  \preauthor{\centering\large\emph}
  \postauthor{\par}
  \predate{\centering\large\emph}
  \postdate{\par}
  \date{14 August 2016}



% Redefines (sub)paragraphs to behave more like sections
\ifx\paragraph\undefined\else
\let\oldparagraph\paragraph
\renewcommand{\paragraph}[1]{\oldparagraph{#1}\mbox{}}
\fi
\ifx\subparagraph\undefined\else
\let\oldsubparagraph\subparagraph
\renewcommand{\subparagraph}[1]{\oldsubparagraph{#1}\mbox{}}
\fi

\begin{document}
\maketitle
\begin{abstract}
This is the abstract. It consists of two paragraphs.
\end{abstract}

\section{Introduction}\label{introduction}

In the United States, annual costs from natural disasters have exceeded
\$35 billion in nine of the last ten years. Extreme events like
Hurricane Katrina and Super Storm Sandy can cause property damage in
excess of \$100 billion and tens or hundreds of billions of dollars in
lost economic output. Moreover, economic costs and loss of life from
natural disasters is likely to rise in the future, as population and
asset values rise, and as climate change leads to more frequent and more
severe disasters (Field 2012). While most disasters are unlikely to have
major economic implications for the national econonomy, they can have
major economic impacts at the regional level, especially in poor
communities (Kousky 2014).

The economic costs of by natural disasters are generally related to the
destruction of homes, plants and equipment, housing and infrastructure,
and the accompanying loss of employment opportunities (Rose 2004; Rose,
n.d.). Such impacts may lead to migration away from the afflicted area.
For example, if a resident's home is destroyed by a disaster he has less
incentive to stay in the area. Destruction of capital can weaken local
economies. If a factory is destroyed, for example, the owner of the
business may choose to shut down or relocate operations rather than
rebuild in the same community. This results in indirect effects to the
local economy as wages and employment decline (\textbf{add some cites
here}), which might induce residents to migrate to areas with better job
prospects. Finally, disasters can injure or kill residents. A disaster
might provide additional information about the riskiness of an area,
leading residents to migrate away from the area toward regions that are
safer.

Outflow migration after a disaster is not only a consequence of a
disaster, but also an important tool to mitigate potential damages from
future disasters (Gráda and O'Rourke 1997). To the extent that migration
following a disaster moves people from relatively high-risk regions to
relatively low-risk regions, future disasters will be less damaging as a
result.

Migration in response to disasters has been well-documented in the case
of major events Hurricane Katrina (Levine, Esnard, and Sapat 2007;
Vigdor 2007). Little attention has been paid to the margin on which
relatively small disasters induce migration. To some extent, small
disasters are the margin on which migration can reduce disaster damages
the most. A powerful hurricane hitting a large city is certain to cause
widespread devastation, and it is unlikely that major cities will
depopulate enough to avoid such damages. Rural communities that are
frequently struck by smaller disasters, on the other hand, are less
populous and it is more likely that migration can reduce the exposure of
people and assets to disaster by a larger proportion.

We investigate the effect of disasters on migration utilizing a model in
which outflow migration at the county-level is driven by county-level
disasters and also affected by attractiveness features of the
destination. Because migration and disasters are both spatial
phenomenon, we employ a spatial econometric model in which outflow
migration can be correlated with outflow migration in neighboring
counties and with neighboring disasters. We use data on county-level
outflow migration for 3,143 counties between 1992 and 2013, as well as
data on \textbf{length(unique(countydamages\_events\$Group.3))} damaging
natural disasters.

We contribute to the disaster-induced migration literature in a number
of ways. Our paper is the first empirical estimate of disaster-induced
migration in the U.S. that utilizes spatial econometric modeling
methods. We also provide a comprehensive, national assessment of
migration following disasters of all sizes and types, while most
previous studies have focused on a single large event. Finally, by
observing the destination to which disaster-stricken populations migrate
we inform the long-run susceptibility of the U.S. population to
disasters.

\section{Disasters and Migration}\label{disasters-and-migration}

Migration following disasters is well documented, particularly in the
case of major events such as Hurricane Katrina or the Fukushima Nuclear
Meltdown in Japan (Landry et al. 2007; Groen and Polivka 2010; Oda
2011). These disasters are particularly well studied because the
magnitude of the disaster induced substantial migration, evacuations,
and quarantines.

Perhaps the most obvious mechanism through which disasters induce
migration is either mandatory or voluntary evacuations. More than 1.5
million people evacuated before Hurricane Katrina, for example, and
several weeks later when Hurricane Rita approached Texas, nearly 3
million people evacuated. As the length of time between the evacuation
and the potential for return grows, it becomes more likely that evacuees
will become permanent migrants (\textbf{add cites}).

Such migration carries important equity implications. Groen and Polivka
(2010) examine the return of migrants following Hurricane Katrina and
find substantial socio-demographic variation in the type of person who
returned to New Orleans following evacuation. They broadly found that
whites were more likely than blacks to return, and those residents with
low income and education were less likely to return than relatively
wealthier and educated evacuees. This is consistent with S. Danziger and
Danziger (2006), who found that the deleterious effects of disasters
fall disproportionately on low-income residents. Similarly, Myers,
Slack, and Singelmann (2008) find that areas with a relatively high
proportion of low-income residents are more likely to experience outflow
migration after disasters than relatively wealthy areas, although this
does not allow for the possibility that disasters lead the wealthiest
residents of a poor area to migrate, while the poor are unable to move.

In addition to disasters directly inducing migration, disasters can also
weaken the local economy, thereby inducing further migration. In the
case of the Tohuku earthquake/tsunami and the ensuring Fukushima nuclear
reactor meltdown, for example, Higuchi et al. (2012) find that
unemployment and underemployment remained high in the affected
prefectures following the disaster. Similarly, they report substantial
increases in outflow migration, although they do not attempt to identify
the direction of causality. Venn (2012) documents not only the effect of
large-scale natural disasters on local labor markets but also examines
the range of government policies that have been undertaken to support
local economies following disasters.

Natural disasters may also provide information to residents about the
likelihood and severity of natural disasters, leading residents to
update their beliefs about the risks of living in an area. If a disaster
leads to an increase in the expected cost of natural disasters in an
area, residents may choose to migrate to avoid future losses. Following
Hurricane Katrina and Hurricane Rita, evacuees overestimated the
likelihood of a subsequent hurricane, and risk perception influenced
their stated preference to return to New Orleans (Baker et al. 2009).
This path may be unlikely, however, as studies on hurricane evacuation
behavior tend to suggest that previous disaster experience does not
appear to drive evacuation behavior (See \textbf{add cites here}).

Disasters do not induce migration with certainty, however. Paul (2005)
finds no evidence of outflow migration following a major tornado in
Bangladesh. He further notes that post-disaster aid counteracts the push
to migrate following a disaster. Gray and Mueller (2012) find only
modest effects of flooding on migration and Halliday (2006) actually
finds that an earthquake in El Salvador reduced outflow migration rather
than increasing it.\footnote{Halliday suggests that this could be an
  increase in the need for laborers to stay at homd to rebuild, or that
  it could suggest a reduction in financial ability to migrate.}

While most studies of migration and disasters are related to a single
disaster, there are a few studies that examine disaster-induced
migration regionally, without focusing on a single dominating disaster.
Bohra-Mishra, Oppenheimer, and Hsiang (2014) follow more than 7,000
Indonesian households and find substantial migration in response to
temperature variation and minor response to variation in rainfall.
Surprisingly, they do not find that disasters result in appreciable
migration. Saldaña-Zorrilla and Sandberg (2009) use a spatial
econometric model to estimate outflow migration from over 2,000
municipalities in Mexico, and find that regions that are more often
affected by disasters have higher migration rates, and that migration is
more likely for an educated individual than an uneducated person. They
also document the effect of local economic conditions - as measured by
crop prices - on migration, although this does not allow for a
relationship between disasters and local economic conditions. Because
the destination of migrants matters in assessing future disaster
exposure, it is also important to note that Saldaña-Zorrilla and
Sandberg (2009) focus only on migration outflows and are unable to
delineate the destination of migrants.

While our paper focuses on migration in response to natural disasters,
there is a related literature that focuses on migration in response to
wars and human conflict. Lozano-Gracia et al. (2010) find that
conflict-related migration tends to be toward relatively safer regions,
although the pull effect toward safer regions may be muted in the case
of natural disasters because human conflict is an ongoing risk
disaster-migration decisions are generally made after the immediate
threat has subsided.

\section{Models of Migration}\label{models-of-migration}

Econometric studies of human migration tend to rely on a gravity model
(Greenwood 1985; Borjas 1989), in which the decision to migrate is
driven by the expected benefits of migration relative to its expected
costs. The benefits of migration are typically improvements in economic
conditions or amenities in the destination relative to the origin, while
the expected costs capture the transporation and relocation expenditures
and are usually modeled as a function of distance. As the difference in
amenity quality between the origin and the destination increases, the
likelihood of migratory flow grows as well. Conversely, as the distance
between the two regions grows, the likelihood of migration falls due to,
for example, travel costs. The attribute that generally matters the most
in migratory gravity models is the wage differential between the origin
and the destination. As economic conditions are better in a prospective
destination relative to the origin, migration is more prevalent.

Simply controlling for distance in the migration model does not
sufficiently account for the spatial correlation in the gravity model,
however, if a region affects its neighbors (Porojan 2001). As a result
it is necessary to explicitly model the spatial dependence in order to
prevent standard errors from being biased. These models tend to follow
LeSage and Pace (2008) in specifying a weights matrix to explicitly
relate neighboring regions. By interacting a spatial weights matrix with
a vector of dependent variables and optionally a matrix of independent
variables, it is possible to estimate migration flows under a wide range
of assumptions about spatial dependence. These models can be estimated
via maximum likelihood and the fit compared using a Lagrange Multiplier
type test.

\section{Data}\label{data}

\subsection{Migration Data}\label{migration-data}

The IRS reports annual, county-level migration based on the change in
location at which tax returns are filed. These reports include not only
those filings that change counties but also the number of claimed
exemptions that change counties and the annual gross adjusted income
associated with the filings. The IRS reports both outflow migration (tax
returns and exemptions that leave a county) as well as inflow migration
(tax returns and exemptions that enter a county), and we compute net
migration by subtracting outflow migration from inflow migration.
Therefore a positive net migration number indicates that on-net people
are moving to a county, while a negative number indicates that on-net
people are leaving a county. The IRS data also includes the number of
filings and exemptions that do not move, providing a base-level
population value that is comparable to the migration data. Figure
\textbf{ref\{mapnum\}} presents the portion of each county's population
that migrated in 2013.

In addition to the county-level migration in-flow and out-flow data, the
IRS also reports migration between pairs of origin and destination
counties. Like the aggregated data, this information is derived from
changes in the filing location of tax returns. In order to preserve
anonymity of specific migrating individuals, the IRS only reports the
number of migrants between origin-destination pairs if there are more
than \textbf{XX} migrants that migrated between a given origin and
destination pair. As a result, migration from particularly
low-population counties and migration toward low-population counties are
relatively less likely to appear in the dataset than migration between
high-population counties.

Finally, we note that between 2011 and 2013 the IRS reported out-flow
and in-flow migration at the county-level stratified by income.

There are several important considerations surrounding the
interpretation of these data. First, the IRS migration data contains
only information on filers and exemptions. Households that did not file
an income tax return are excluded from the sample. If one expects that
the unemployed or seniors who live off of Social Security or other
retirement benefits are disproportionately more or less likely to move
than workers, our results will be biased. Filers are slightly more
likely than non-filers to migrate (Molloy, Smith, and Wozniak 2011).
Similarly, because the IRS measurement of residence is determined by
filing location, we are unable to observe temporary migration if the
migrant does not live in the destination region long enough to be
required to file a tax return. The IRS migration data conveys the
benefit, though, that we are able to differentiate medium and long term
migration from evacuation.

\subsection{Disaster Data}\label{disaster-data}

The National Oceanographic and Atmospheric Administration (NOAA) Storm
Events Database contains data on a wide range of meteorological
disasters. For each disaster in the database, NOAA reports crop damage,
property damage, and deaths at either the county-level or at the NOAA
zone-level. Zones are generally smaller than counties and usually do not
overlap county lines. In these instances, we aggregate damages up to the
county-level. When NOAA zones overlap county lines, we disaggregate the
damages across counties based on the areal proportion of the zone that
overlaps each county.

NOAA categorizes disasters into 73 categories. Table
\textbf{ref\{categorylist\}} lists the ten NOAA disaster types that
resulted in the most property damage between 2005 and 2015, as well as
the corresponding crop damage and deaths for these events. The most
damaging disaster category is storm surges and extreme tides, although
these disasters resulted in a relatively low number of deaths.
Hurricanes and tornados have each resulted in approximately 1,000 deaths
between 2005 and 2015, as well as substantial amounts of property and,
in the case of hurricanes, crop damage. Some disasters, such as those
that are related to extreme wind, result in a relatively high number of
fatalities but only moderate property and crop damage. Because our
migration data are available on the annual basis, we aggregate damages
for each county to the annual level.

\section{Methodology}\label{methodology}

We present two models of migration, one focusing on net migration from
each county and the other examining migration between pairs of
counties.\footnote{These will likely result in two papers but for now we
  lay out the outline of both approaches.} The net migration models
focuses on the total number of people entering or leaving a county,
regardless of their destination. This avoids some data censoring in
cases in which county-to-county flows are censored to maintain
anonymity. The county-to-county approach allows much more focus on the
characteristics of the destination that influence migration, although it
is computationally much more complex than the net migration models.

\subsection{Net Migration}\label{net-migration}

Our basic model involves estimating the net migration from a county
based on natural disaster damages. We then proceed by adding increasing
layers of spatial and temporal specificity to the estimating equation.

Our primary estimating equation is:

\begin{equation} \label{olsequation}
netmig =  damage\beta + population \gamma + X \delta +\epsilon 
\end{equation}

where \(netmig\) is the net migration from a county, damage is the
magnitude of disaster damages affecting that county, population is the
county's population and X is a matrix of additional explanatory
variables that could affect migration in a county. The explanatory
variable matrix \(X\) can include attributes such as unemployment
levels, income, or racial composition. In the basic model, \(\epsilon\)
is an iid error term. Our variable of interest \(\beta\) measures how
contemporaneous natural disaster damages affect migration away from a
county while \(\gamma\) controls for the size of the county prior to the
migration.

Next, we note that there may be a spatial component associated with
outflow migration (\textbf{cites here}). A county that experiences
substantial outflow migration due to, for example, a weak local economy
is likely to be geographically close to other counties that have weak
local economies.

We therefore respecify the error term as:

\begin{equation} \label{spatialerrorcomponent}
\epsilon = \lambda W \epsilon + \eta
\end{equation}

where W is a spatial weights matrix that relates the counties in our
sample to one another so that \(\lambda W \epsilon\) is the spatial
portion of the error term that is driven by nearby counties' errors, and
\(\eta\) is the idiosyncratic error term in the traditional sense. We
substitute equation \ref{spatialerrorcomponent} re-arrange equation
\ref{olsequation} ,and re-estimate.

\begin{equation} \label{spatialerrorequation}
netmig = \rho W netmig +  damage\beta + population \gamma + X \delta +\epsilon .
\end{equation}

Note that in this estimating equation net migration in each county
depends explicitly on the net migration in nearby counties, as defined
by the spatial weights matrix, \(W\).

In order to examine the effect of income on the ability to migrate
following a disaster, we next estimate the effect of disasters on net
migration while differentiating by income strata. In particular, we
estimate:

\begin{equation} \label{incomeequation}
netmig^k =  damage\beta^k + population^k \gamma^k + X^k \delta^k +\epsilon 
\end{equation}

which is identical to equation \ref{olsequation} except for the
introduction of the superscript k which indicates the income strata.
This equation estimates a separate coefficient, \(\beta^k\), for each of
the income strata, indicating the effect of damages on net migration
among individuals in income strata \(k\). Note that \(damage\) is not
differentiated by income strata.

\subsection{Origin-Destination Models}\label{origin-destination-models}

Finally, in order to examine the destination of disaster-related
migrants, we specify an origin-destination model in which we estimate
the migration from county \(j\) to county \(k\) as

\begin{equation}
  mig_{jk} = damage_j \beta_o + damage_k \beta_d + X_j \gamma_o + X_k \gamma_d + \epsilon .
\end{equation}

\(mig_{jk}\) is the flow of migration from county \(j\) to county \(k\),
and \(\beta_o\) and \(\beta_d\) capture the effect of disaster damage in
the origin and destination counties, respectively, on the migration
decision. The matrices \(X_j\) and \(X_k\) contain additional
explanatory variables at the origin and destination, such as cost of
living, unemployment rates, racial composition and local amenities. The
perceived safety vulnerability in the destination location can either be
captured by the \(damage_k\) term or included in the \(X_k\) matrix.

\section{Results}\label{results}

To be completed.

\section{Conclusion}\label{conclusion}

To be completed.

\section{Tables and Figures}\label{tables-and-figures}

Not Sure.

\newpage

\section*{References}\label{references}
\addcontentsline{toc}{section}{References}

\hypertarget{refs}{}
\hypertarget{ref-baker2009explaining}{}
Baker, Justin, W Douglass Shaw, David Bell, Sam Brody, Mary Riddel,
Richard T Woodward, and William Neilson. 2009. ``Explaining Subjective
Risks of Hurricanes and the Role of Risks in Intended Moving and
Location Choice Models.'' \emph{Natural Hazards Review} 10 (3). American
Society of Civil Engineers: 102--12.

\hypertarget{ref-bohra2014nonlinear}{}
Bohra-Mishra, Pratikshya, Michael Oppenheimer, and Solomon M Hsiang.
2014. ``Nonlinear Permanent Migration Response to Climatic Variations
but Minimal Response to Disasters.'' \emph{Proceedings of the National
Academy of Sciences} 111 (27). National Acad Sciences: 9780--5.

\hypertarget{ref-borjas1989economic}{}
Borjas, George J. 1989. ``Economic Theory and International Migration.''
\emph{International Migration Review}. JSTOR, 457--85.

\hypertarget{ref-danziger2006poverty}{}
Danziger, Sheldon, and Sandra K Danziger. 2006. ``Poverty, Race, and
Antipoverty Policy Before and After Hurricane Katrina.'' \emph{Du Bois
Review} 3 (01). Cambridge Univ Press: 23--36.

\hypertarget{ref-field2012managing}{}
Field, Christopher B. 2012. \emph{Managing the Risks of Extreme Events
and Disasters to Advance Climate Change Adaptation: Special Report of
the Intergovernmental Panel on Climate Change}. Cambridge University
Press.

\hypertarget{ref-grada1997migration}{}
Gráda, Cormac Ó, and Kevin H O'Rourke. 1997. ``Migration as Disaster
Relief: Lessons from the Great Irish Famine.'' \emph{European Review of
Economic History} 1 (1). Oxford University Press: 3--25.

\hypertarget{ref-gray2012natural}{}
Gray, Clark L, and Valerie Mueller. 2012. ``Natural Disasters and
Population Mobility in Bangladesh.'' \emph{Proceedings of the National
Academy of Sciences} 109 (16). National Acad Sciences: 6000--6005.

\hypertarget{ref-greenwood1985human}{}
Greenwood, Michael J. 1985. ``Human Migration: Theory, Models, and
Empirical Studies.'' \emph{Journal of Regional Science} 25 (4). Wiley
Online Library: 521--44.

\hypertarget{ref-groen2010going}{}
Groen, Jeffrey A, and Anne E Polivka. 2010. ``Going Home After Hurricane
Katrina: Determinants of Return Migration and Changes in Affected
Areas.'' \emph{Demography} 47 (4). Springer: 821--44.

\hypertarget{ref-halliday2006migration}{}
Halliday, Timothy. 2006. ``Migration, Risk, and Liquidity Constraints in
El Salvador.'' \emph{Economic Development and Cultural Change} 54 (4).
JSTOR: 893--925.

\hypertarget{ref-higuchi2012impact}{}
Higuchi, Yoshio, Tomohiko Inui, Toshiaki Hosoi, Isao Takabe, and Atsushi
Kawakami. 2012. ``The Impact of the Great East Japan Earthquake on the
Labor Market: Need to Resolve the Employment Mismatch in the
Disaster-Stricken Areas.'' \emph{Japan Labor Review} 9 (4): 4--21.

\hypertarget{ref-kousky2014informing}{}
Kousky, Carolyn. 2014. ``Informing Climate Adaptation: A Review of the
Economic Costs of Natural Disasters.'' \emph{Energy Economics} 46.
Elsevier: 576--92.

\hypertarget{ref-landry2007going}{}
Landry, Craig E, Okmyung Bin, Paul Hindsley, John C Whitehead, and
Kenneth Wilson. 2007. ``Going Home: Evacuation-Migration Decisions of
Hurricane Katrina Survivors.'' \emph{Southern Economic Journal}. JSTOR,
326--43.

\hypertarget{ref-lesage2008spatial}{}
LeSage, James P, and R Kelley Pace. 2008. ``Spatial Econometric Modeling
of Origin-Destination Flows.'' \emph{Journal of Regional Science} 48
(5). Wiley Online Library: 941--67.

\hypertarget{ref-levine2007population}{}
Levine, Joyce N, Ann-Margaret Esnard, and Alka Sapat. 2007. ``Population
Displacement and Housing Dilemmas Due to Catastrophic Disasters.''
\emph{Journal of Planning Literature} 22 (1). Sage Publications: 3--15.

\hypertarget{ref-lozano2010journey}{}
Lozano-Gracia, Nancy, Gianfranco Piras, Ana Maria Ibáñez, and Geoffrey
JD Hewings. 2010. ``The Journey to Safety: Conflict-Driven Migration
Flows in Colombia.'' \emph{International Regional Science Review} 33
(2). SAGE Publications: 157--80.

\hypertarget{ref-molloy2011internal}{}
Molloy, Raven, Christopher L Smith, and Abigail Wozniak. 2011.
``Internal Migration in the United States.'' \emph{The Journal of
Economic Perspectives} 25 (3). American Economic Association: 173--96.

\hypertarget{ref-myers2008social}{}
Myers, Candice A, Tim Slack, and Joachim Singelmann. 2008. ``Social
Vulnerability and Migration in the Wake of Disaster: The Case of
Hurricanes Katrina and Rita.'' \emph{Population and Environment} 29 (6).
Springer: 271--91.

\hypertarget{ref-oda2011grasping}{}
Oda, Takashi. 2011. ``Grasping the Fukushima Displacement and
Diaspora.'' The.

\hypertarget{ref-paul2005evidence}{}
Paul, Bimal Kanti. 2005. ``Evidence Against Disaster-Induced Migration:
The 2004 Tornado in North-Central Bangladesh.'' \emph{Disasters} 29 (4).
Wiley Online Library: 370--85.

\hypertarget{ref-porojan2001trade}{}
Porojan, Anca. 2001. ``Trade Flows and Spatial Effects: The Gravity
Model Revisited.'' \emph{Open Economies Review} 12 (3). Springer:
265--80.

\hypertarget{ref-rose2004economic}{}
Rose, Adam. 2004. ``Economic Principles, Issues, and Research Priorities
in Hazard Loss Estimation.'' In \emph{Modeling Spatial and Economic
Impacts of Disasters}, 13--36. Springer.

\hypertarget{ref-rose2016earthquake}{}
---------. n.d. ``Economic Aspects of Earthquake Lsos Estimation.'' In
\emph{Handbook of Earthquake Engineering}. CRC Press.

\hypertarget{ref-saldana2009impact}{}
Saldaña-Zorrilla, Sergio O, and Krister Sandberg. 2009. ``Impact of
Climate-Related Disasters on Human Migration in Mexico: A Spatial
Model.'' \emph{Climatic Change} 96 (1-2). Springer: 97--118.

\hypertarget{ref-venn2012helping}{}
Venn, Danielle. 2012. ``Helping Displaced Workers Back into Jobs After a
Natural Disaster.'' OECD Publishing.

\hypertarget{ref-vigdor2007katrina}{}
Vigdor, Jacob L. 2007. ``The Katrina Effect: Was There a Bright Side to
the Evacuation of Greater New Orleans?'' \emph{The BE Journal of
Economic Analysis \& Policy} 7 (1).

\end{document}
